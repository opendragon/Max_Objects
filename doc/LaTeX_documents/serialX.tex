\ProvidesFile{serialX.tex}[v1.1.4]
\startObject{\objNameS{serialX}}{serialX}
\index{Themes!Device~interface!serialX}
\objPicture{serialXsymbol.ps}
\objItemDescription{\objNameD{serialX} is an interface to the serial ports on a Macintosh.
It provides functionality over and above that of the standard \objNameS{serial} object.}

\objItemCreated{June 1998}

\objItemVersion{1.1.4}

\objItemHelp{yes}

\objItemTheme{Device interface}

\objItemClass{Devices}

\objItemArgs{\nothing}

  \objListArgBegin
  \objListArgItem{port}{symbol}{either `modem', `printer', or a single letter from `a' to `z' which
        represents the port to be connected to}

  \objListArgItem{baud-rate}{(optional) integer\textnormal{/}symbol}{the baud rate to set the port to.
        The acceptable values are: 150, 300, 600, 1200, 1800, 2400, 3600, 4800, 7200, 9600, 14400,
        19200, 28800, 38400, 57600, 115200 and 230400.
        Some of the baud rates can also be expressed in terms of `kilobaud': 0.3k (or .3k), 0.6k
        (or .6k), 1.2k, 1.8k, 2.4k, 3.6k, 4.8k, 7.2k, 9.6k, 14.4k, 19.2k, 28.8k, 38.4k, 57.6k,
        115.2k or 230.4k.
        The following special externally clocked rates may be available, if the serial port supports them:
        MIDI\_1, MIDI\_16, MIDI\_32 and MIDI\_64.
        These correspond to rate multipliers of 1, 16, 32 and 64, respectively.
        Use of these special rates forces the data bits to 8 and the stop bits to 1.
        Note that the character `k' is lower case.
        The default baud rate is 4800.}

  \objListArgItem{data-bits}{(optional) integer}{the number of data bits to set the port to.
        The value is between 5 and 9, inclusive.
        The default number of data bits is 8.}

  \objListArgItem{stop-bits}{(optional) float}{the number of stop bits to set the port to.
        The acceptable values are: 1, 1.5 or 2.
        The default number of stop bits is 1.}

  \objListArgItem{parity}{(optional) symbol}{the parity mode to set the port to.
        The acceptable values are: `off'/'no'/'none', `even' or `odd'.
        The default parity mode is `off'.}

  \objListArgItem{chunks}{(optional) integer}{the number of characters in each chunk that will be
        sent when `chunk' mode is active.
        The default chunk size is 10 and the maximum is 80.}

  \objListArgEnd

\objItemInlet{\nothing}

  \objListIOBegin
  \objListIOItem{integer\textnormal{/}list\textnormal{/}bang}{the command input}
  \objListIOEnd

\objItemOutlet{\nothing}

  \objListIOBegin
  \objListIOItem{integer\textnormal{/}list}{returned characters (as single characters,
        if `chunk' mode is inactive, or as lists of characters, if `chunk' mode is active) from
        the port of the serial device}

  \objListIOItem{bang}{a break signal was seen on the port of the serial device}

  \objListIOItem{bang}{the break operation has completed}

  \objListIOEnd

\objItemCompanion{none}

\objItemStandalone{yes}

\objItemRetainsState{yes, the serial port settings}

\objItemCompatibility{\MaxName{} 3.x and \MaxName{} 4.x \{OS 9 only\}}

\objItemFat{Fat}

\objItemCommands[]

  \objListCmdBegin

  \objListCmdItem{\emphcorr{bang}}{}
  Return any characters received from the port of the serial device (since the last time a
  \objCmdQ{bang} command was received), in the form of one or more lists of characters, if
  `chunk' mode is active, or as single characters, if `chunk' mode is inactive.
  Any detected `break' signals from the port of the serial device will also be reported.

  \objListCmdItem{baud}{integer\textnormal{/}symbol}
  Change the baud rate of the port to the given value.
  The acceptable values are: 150, 300, 600, 1200, 1800, 2400, 3600, 4800, 7200, 9600, 14400,
  19200, 28800, 38400, 57600, 115200 and 230400.
  Some of the baud rates can also be expressed in terms of `kilobaud': 0.3k (or .3k), 0.6k
  (or .6k), 1.2k, 1.8k, 2.4k, 3.6k, 4.8k, 7.2k, 9.6k, 14.4k, 19.2k, 28.8k, 38.4k, 57.6k,
  115.2k or 230.4k.
  The following special externally clocked rates may be available, if the serial port supports them:
  MIDI\_1, MIDI\_16, MIDI\_32 and MIDI\_64.
  These correspond to rate multipliers of 1, 16, 32 and 64, respectively.
  Use of these special rates forces the data bits to 8 and the stop bits to 1.
  Note that the character `k' is lower case.

  \objListCmdItem{break}{}
  Send a `break' signal out the port of the serial device.

  \objListCmdItem{chunk}{\textnormal{[}on\textnormal{/}off\textnormal{]}}
  Chunking of received data will be enabled (`on'), disabled (`off') or reversed, if no argument
  is given.
  The chunk size is established when the \objNameX{serialX} object is created.

  \objListCmdItem{dtr}{\textnormal{[}assert\textnormal{/}negate\textnormal{/}on\textnormal{/}off\textnormal{]}}
  The DTR signal will asserted (`assert'), negated (`negate') or used for handshaking
  (`off' or `on' or no argument).
  When no argument is given, the use of the DTR signal for handshaking will be toggled.
  The DTR handshake is set active when the \objNameX{serialX} object is created.

  \objListCmdItem{\emphcorr{float}}{}
  Send the single character that is the ASCII equivalent to the given value, out the port of
  the serial device.

  \objListCmdItem{\emphcorr{integer}}{}
  Send the single character that is the ASCII equivalent to the given value, out the port of
  the serial device.

  \objListCmdItem{list}{anything}
  Send the contents of the list out the port of the serial device.
  Integer and float values are sent as the equivalent ASCII character and symbols are sent as
  a string of ASCII characters.
  Note that no white space is introduced between values, so the command \objCmdQ{1 2 3 alpha 4}
  results in the characters `123alpha4' being sent out the port of the serial device.

  \objListCmdItem{text}{anything}
  Send the characters corresponding to the arguments to the command out the port of the serial
  device.
  Note that no white space is introduced between values, so the command \objCmdQ{text 1 2 3 4}
  results in the characters `1234' being sent out the port of the serial device.
  
  \objListCmdEnd

\objItemFile

\objItemMessage

\objItemComments[The \objNameX{serialX} object was designed to address some weaknesses of the standard
\objNameS{serial} object--the `break' signal was neither reported nor generated,
the maximum baud rate was well below what most serial ports can handle, and output was always
in the form of single characters.
\objNameX{serialX} was originally developed as a companion for the \objName{gvp} object
(which needed to be able to send a `break' signal), and was extended to support higher baud rates
and `chunking' for the \objName{mtc} object.]

\objEnd{\objNameE{serialX}}

% $Log: serialX.tex,v $
% Revision 1.5  2006/07/20 04:47:54  churchoflambda
% Re-added the files to record their changes.
%
% Revision 1.3  2005/08/02 15:07:09  churchoflambda
% Added CVS tags; add rail diagrams for pfsm, map1d, map2d, map3d and listen.
%
