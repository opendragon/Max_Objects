\ProvidesFile{spaceball.tex}[v1.0.3]
\startObject{\objNameS{spaceball}}{spaceball}
\index{Themes!Device~interface!spaceball}
\objPicture{spaceballsymbol.ps}
\objItemDescription{\objNameD{spaceball} is an interface to the \companyReference{http://www.labtec.com}{Labtec}
six-degrees-of-freedom trackball, the Spaceball.
It sends commands to a \objReference{serialX} object, which controls the serial port that the Spaceball is attached to, and responds to data returned
from the Spaceball via the \objName{serialX} object.}

\objItemCreated{July 2001}

\objItemVersion{1.0.3}

\objItemHelp{no}

\objItemTheme{Device interface}

\objItemClass{Devices}

\objItemArgs{\ }

  \objListArgBegin
  
  \objListArgItem{mode}{(optional) symbol}{the initial processing mode (additive (`add') or differential (`delta')) that is to be
       used.
       The default mode is additive.}
  
  \objListArgItem{poll-rate}{(optional) integer}{the rate (in milliseconds) at which the companion \objName{serialX} object is polled
       via a sample request.
       The default rate is 100 milliseconds between sample requests.}

  \objListArgEnd

\objItemInlet{\ }

  \objListIOBegin
  \objListIOItem{list}{the command channel}
  
  \objListIOItem{anything}{the output of the companion \objName{serialX} object}
  
  \objListIOEnd

\objItemOutlet{\ }

  \objListIOBegin
  \objListIOItem{list}{result data from the Spaceball,
  in the form of a button list (a three-element list, starting with the symbol `button',
  followed by the button number and the button state, where `0' is up and `1' is down, for each button transition),
  a rotation list (a four-element list, starting with the symbol `rotate',
  with the cumulative rotation factors as three floating point numbers, in the order `X', `Y', `Z')
  or a translation list (a four-element list, starting with the symbol `translate',
  with the cumulative translation terms as three floating point numbers, in the order `X', `Y', `Z').
  Moving the sensor ball results in both a rotation list and a translation list;
  pressing a button results in two button lists, the first when the button is pressed and the second when it is released}
  
  \objListIOItem{bang}{the sample request to send to the companion \objName{serialX} object}

  \objListIOItem{anything}{the data to send to the companion \objName{serialX} object}

  \objListIOItem{bang}{the \objCmdQ{chunk} request to send to the companion \objName{serialX} object, via a message object}
  
  \objListIOItem{bang}{an error was detected}
  
  \objListIOEnd

\objItemCompanion{works with \objName{serialX} objects (not \objNameS{serial})}

\objItemStandalone{yes}

\objItemRetainsState{yes}

\objItemCompatibility{\MaxName{} 3.x and \MaxName{} 4.x \{OS 9 and OS X\}}

\objItemFat{Fat}

\objItemCommands[]

  \objListCmdBegin
  
  \objListCmdItem{mode}{symbol}
  Change the processing mode (where \objCmdArg{symbol} is `add' or `delta') that is to be used.

  \objListCmdItem{reset}{}
  Send a device reset sequence to the Spaceball and reset the rotation and translation vectors.

  \objListCmdItem{verbose}{\textnormal{[}on\textnormal{/}off\textnormal{]}}
  Tracing to the \MaxName{} window of the messages sent will be enabled (`on'), disabled (`off') or reversed, if no argument is given.
  
  \objListCmdItem{zero}{}
  Reset the rotation and translation vectors.

  \objListCmdEnd

\objItemFile

\objItemMessage

\objItemComments[Figure~\objImageReference{diagram:spaceballconnect} shows how to connect an \objNameX{spaceball} object to a
\objName{serialX} object.]
\objDiagram{spaceballconnections.ps}{spaceballconnect}{Connecting a \objNameX{spaceball} object to a \objName{serialX} object}

\objEnd{\objNameE{spaceball}}
