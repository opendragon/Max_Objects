\ProvidesFile{Vtake.tex}[v1.0.4]
\startObject{\objNameS{Vtake}}{Vtake}
\index{Themes!Vector~manipulation!Vtake}
\index{Vectors!Dyadic~operations!Vtake}
\objPicture{Vtakesymbol.ps}
\objItemDescription{\objNameD{Vtake} is an implementation of the \compLang{APL} `take' operator
($\uparrow$)\index{APL!take}, which is used to return leading or trailing elements of a vector
(in \MaxName{} terms, a list).}

\objItemCreated{July 2000}

\objItemVersion{1.0.4}

\objItemHelp{yes}

\objItemTheme{Vector manipulation}

\objItemClass{Lists}

\objItemArgs{\nothing}

  \objListArgBegin
  \objListArgItem{how-many}{integer}{the number of elements to take.
      A positive number indicates that the elements are taken from the beginning of the input list,
      while a negative number indicates that the elements are to be taken from the end of the list.}
  \objListArgEnd

\objItemInlet{\nothing}

  \objListIOBegin
  \objListIOItem{bang\textnormal{/}list}{the list to reduce}

  \objListIOItem{integer}{the number of elements to take.
     This replaces the initial argument.}
  \objListIOEnd

\objItemOutlet{\nothing}

  \objListIOBegin
  \objListIOItem{list}{the reduced list, or the previous result (if a `bang' is received)}
  \objListIOEnd

\objItemCompanion{none}

\objItemStandalone{yes}

\objItemRetainsState{yes, the number of elements to take}

\objItemCompatibility{\MaxName{} 3.x and \MaxName{} 4.x \{OS 9 and OS X\}}

\objItemFat{Fat}

\objItemCommands[]

  \objListCmdBegin
  \objListCmdItem{\emphcorr{bang}}{}
  Return the previous result, if any.
  \objListCmdEnd

\objItemFile

\objItemMessage

\objItemComments

\objEnd{\objNameE{Vtake}}

% $Log: Vtake.tex,v $
% Revision 1.3  2005/08/02 15:07:08  churchoflambda
% Added CVS tags; add rail diagrams for pfsm, map1d, map2d, map3d and listen.
%
