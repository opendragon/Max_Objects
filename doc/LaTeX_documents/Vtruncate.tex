\ProvidesFile{Vtruncate.tex}[v1.0.5]
\startObject{\objNameS{Vtruncate}}{Vtruncate}
\index{Themes!Miscellaneous!Vtruncate}
\index{Themes!Vector~manipulation!Vtruncate}
\index{Vectors!Monadic~operations!Vtruncate}
\objPicture{Vtruncatesymbol.ps}
\objItemDescription{\objNameD{Vtruncate} calculates the integer part of the value given (either a
list or a single number).}

\objItemCreated{November 2000}

\objItemVersion{1.0.5}

\objItemHelp{yes}

\objItemTheme{Miscellaneous}

\objItemClass{Arith/Logic/Bitwise, Lists}

\objItemArgs{\ }

  \objListArgBegin
  \objListArgItem{mode}{(optional) symbol}{either `f', `i' or `m' to indicate whether the output
        is to be floating-point values only, integer values only, or mixed values.
        Mixed values are floating-point if the input is floating-point and integer if the input is
        integer.
        The default is `m'.}
  
  \objListArgEnd

\objItemInlet{\ }

  \objListIOBegin
  \objListIOItem{anything}{the input to be processed}
  \objListIOEnd

\objItemOutlet{\ }

  \objListIOBegin
  \objListIOItem{anything}{the input after removing the fractional part, or the previous result
     (if a `bang' is received)}
  \objListIOEnd

\objItemCompanion{none}

\objItemStandalone{yes}

\objItemRetainsState{no}

\objItemCompatibility{\MaxName{} 3.x and \MaxName{} 4.x \{OS 9 and OS X\}}

\objItemFat{Fat}

\objItemCommands[]

  \objListCmdBegin
  \objListCmdItem{\emph{bang}}{}
  Return the previous result, if any.
  \objListCmdEnd

\objItemFile

\objItemMessage

\objItemComments

\objEnd{\objNameE{Vtruncate}}
