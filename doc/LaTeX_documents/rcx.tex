\ProvidesFile{rcx.tex}[v1.0.4]
\startObject{\objNameS{rcx}}{rcx}
\index{Themes!Device~interface!rcx}
\objPicture{rcxsymbol.ps}
\objItemDescription{\objNameD{rcx} is an interface to the \companyReference{http://www.lego.com}{LEGO} MINDSTORMS RCX device.
It uses the Ghost API from LEGO to connect to the device, using either a USB or serial port.
Note that it does not use the \objName{serialX} or \objNameS{serial} objects for this communication.}

\objItemCreated{April 2002}

\objItemVersion{1.0.4}

\objItemHelp{no}

\objItemTheme{Device interface}

\objItemClass{Devices}

\objItemArgs{\nothing}

  \objListArgBegin
  \objListArgItem{port-class}{(optional) symbol}{the kind of Ghost port (`usb' or `serial') that is to be used.
       By default, the port-class is USB.}

  \objListArgItem{which-port}{(optional) symbol}{the name of the port within the port-class that is to be used.
       By default, the first available port is used.}

  \objListArgEnd

\objItemInlet{\nothing}

  \objListIOBegin
  \objListIOItem{list}{the command channel}
  \objListIOEnd

\objItemOutlet{\nothing}

  \objListIOBegin
  \objListIOItem{list}{the data requested from the RCX device.
  The individual commands indicate the format of their results.}
  
  \objListIOItem{bang}{the command has completed}

  \objListIOItem{bang}{an error was detected}
  
  \objListIOEnd

\objItemCompanion{none}

\objItemStandalone{yes}

\objItemRetainsState{yes}

\objItemCompatibility{\MaxName{} 3.x and \MaxName{} 4.x \{currently OS 9 only\}}

\objItemFat{PPC-only}

\objItemCommands[]

  \objListCmdBegin
  \objListCmdItem{clearmemory}{}
  Clears all programs out of the RCX.
  
  \objListCmdItem{clearsensor}{index}
  Clear the value of the given sensor, where \objCmdArg{index} is between 1 and 3.
  
  \objListCmdItem{clearsound}{}
  Clear the sound buffer.
  See also the \objCmdQ{mute} command.
  
  \objListCmdItem{continuetask}{index}
  Resume the execution of the given task, where \objCmdArg{index} is between 1 and 10.
  
  \objListCmdItem{floatoutput}{set}
  Set the given set of outputs to ``coast'', where \objCmdArg{set} is between 1 and 7
  (the individual outputs are encoded as 1, 2 and 4, which are then added together to get the set of outputs to change).
  
  \objListCmdItem{getallsensors}{}
  Read the values, types and modes of all three sensors of the RCX and return them as a ten element list, starting with the
  symbol `allsensors', followed by triples (value as an integer, type as one of `nosensor', `switch', `temperature', `reflection',
  `angle' or `unknown' and mode as one of `raw', `boolean', `transition', `periodcount', `percent', `celsius', `fahrenheit',
  `anglestep' or `unknown') for each of the sensors, in order of the sensor number.
  
  \objListCmdItem{getallvariables}{}
  Read the values of all thirty-two variables of the RCX and return them as a list, startimg with the
  symbol `allvariables' and followed by the value of each of the variables.
  
  \objListCmdItem{getbattery}{}
  Read the battery level from the RCX and return it as a two element list, starting with the symbol `battery', followed
  by a floating-point value representing the voltage.
  
  \objListCmdItem{getsensormode}{index}
  Read the mode of the given sensor, where \objCmdArg{index} is between 1 and 3, and return it as a two element list,
  starting with the symbol `sensormode', followed by the current value of the sensor as one of `raw', `boolean', `transition', `periodcount', `percent', `celsius', `fahrenheit',
  `anglestep' or `unknown'.
  
  \objListCmdItem{getsensortype}{index}
  Read the type of the given sensor, where \objCmdArg{index} is between 1 and 3, and return it as a two element list,
  starting with the symbol `sensortype', followed by the current type of the sensor as one of `nosensor', `switch', `temperature',
  `reflection', `angle' or `unknown'.
  
  \objListCmdItem{getsensorvalue}{index}
  Read the value of the given sensor, where \objCmdArg{index} is between 1 and 3, and return it as a two element list,
  starting with the symbol `sensor', followed by the current value of the sensor as an integer.
  
  \objListCmdItem{getslot}{}
  Read the index of the active program and return it as a two element list, starting with the symbol `slot', followed
  by the index as an integer between 1 and 5.
  
  \objListCmdItem{getvariable}{index}
  Read the value of the given variable, where \objCmdArg{index} is between 1 and 32, and return it as a two element list,
  starting with the symbol `variable', followed by the current value of the variable as an integer.
  
  \objListCmdItem{getversion}{}
  Read the version number of the ROM and the firmware from the RCX and return it as a three element list, starting
  with the symbol `version', followed by the ROM version number and the firmware version number, represented as integers.
  
  \objListCmdItem{mute}{}
  Clear the sound buffer of the RCX and ignore future sound requests.
  See also the \objCmdQ{clearsound} command.
  
  \objListCmdItem{playsound}{sound}
  Play the given system sound, which is either `keyclick', `beep', `sweepdown', `sweepup', `error' or `fast\-sweep'.
  
  \objListCmdItem{playtone}{frequency duration}
  Play the given tone, where \objCmdArg{frequency} is in Hertz and \objCmdArg{duration} is in 100ths of a second.
  
  \objListCmdItem{setoutputdirection}{set direction}
  Set the direction of the given set of outputs, where \objCmdArg{set} is between 1 and 7
  (the individual outputs are encoded as 1, 2 and 4, which are then added together to get the set of outputs to change) and
  \objCmdArg{direction} is either `backward', `reverse' or `forward'.
  
  \objListCmdItem{setmessage}{message}
  Set the message buffer to the given value, where \objCmdArg{message} is between 1 and 255.
  
  \objListCmdItem{setoutputpower}{set level}
  Set the power level for the given set of outputs, where \objCmdArg{set} is between 1 and 7
  (the individual outputs are encoded as 1, 2 and 4, which are then added together to get the set of outputs to change) and
  \objCmdArg{level} is between 1 and 8.
  
  \objListCmdItem{setsensormode}{index mode}
  Set the mode of the given sensor, where \objCmdArg{index} is between 1 and 3 and \objCmdArg{mode} is one of `raw', `boolean',
  `transition', `periodcount', `percent', `celsius', `fahrenheit' or `anglestep'.
  
  \objListCmdItem{setsensortype}{index type}
  Set the type of the given sensor, where \objCmdArg{index} is between 1 and 3 and \objCmdArg{type} is one of `nosensor',
  `switch', `temperature', `reflection' or `angle'.
  
  \objListCmdItem{setslot}{index}
  Select the given program slot as the active program, where \objCmdArg{index} is between 1 and 5.
  
  \objListCmdItem{setwatch}{hour minute}
  Set the displayed time on the RCX to the given values.
  
  \objListCmdItem{run}{}
  Start the active program's execution.
  
  \objListCmdItem{setvariable}{index value}
  Set the value of the given variable, where \objCmdArg{index} is between 1 and 32.
  
  \objListCmdItem{startoutput}{set}
  Set the given set of outputs on, where \objCmdArg{set} is between 1 and 7
  (the individual outputs are encoded as 1, 2 and 4, which are then added together to get the set of outputs to change).
  
  \objListCmdItem{starttask}{index}
  Start the execution of the given task, where \objCmdArg{index} is between 1 and 10.
  
  \objListCmdItem{stopalltasks}{}
  Stop the execution of all tasks for the active program on the RCX device.

  \objListCmdItem{stopoutput}{set}
  Set the given set of outputs off, where \objCmdArg{set} is between 1 and 7
  (the individual outputs are encoded as 1, 2 and 4, which are then added together to get the set of outputs to change).
  
  \objListCmdItem{stoptask}{index}
  Stop the execution of the given task, where \objCmdArg{index} is between 1 and 10.
  
  \objListCmdItem{turnoff}{}
  Turn off the RCX device.
  
  \objListCmdItem{unmute}{}
  Re-enable the processing of sound requests.

  \objListCmdItem{verbose}{\textnormal{[}on\textnormal{/}off\textnormal{]}}
  Tracing to the \MaxName{} window of the messages sent will be enabled (`on'), disabled (`off') or reversed, if no argument is given.
  
  \objListCmdEnd

\objItemFile

\objItemMessage

\objItemComments[The \companyReference{http://homepage.mac.com/rbate/MacNQC/index.html}{MacNQC} development environment is recommended for
developing code to be downloaded into the RCX device.]

\objEnd{\objNameE{rcx}}

% $Log: rcx.tex,v $
% Revision 1.5  2005/08/02 15:07:09  churchoflambda
% Added CVS tags; add rail diagrams for pfsm, map1d, map2d, map3d and listen.
%
