\ProvidesFile{udpPort.tex}[v1.0.0]
\startObject{\objNameS{udpPort}}{udpPort}
\index{Themes!TCP/IP!udpPort}
\objPicture{udpPortsymbol.ps}
\objItemDescription{\objNameD{udpPort} is an interface to the UDP/IP stack on a Macintosh,
providing an endpoint to communicate with another \objReference{udpPort} object.}

\objItemCreated{July 2005}

\objItemVersion{1.0.0}

\objItemHelp{yes}

\objItemTheme{TCP/IP}

\objItemClass{Devices}

\objItemArgs{\nothing}

  \objListArgBegin
  \objListArgItem{port}{(optional) integer}{the number of the port to communicate on.
      The default port is 65535, which is also the maximum acceptable port number}
      
  \objListArgItem{buffers}{(optional) integer}{the number of receive buffers to use.
      The default number of buffers is 25}
      
  \objListArgEnd

\objItemInlet{\nothing}

  \objListIOBegin
  \objListIOItem{list}{the command input}
  \objListIOEnd

\objItemOutlet{\nothing}

  \objListIOBegin
  \objListIOItem{list}{the status or response.
      Status messages (triggered by a \objCmdQ{bang} or \objCmdQ{status} command) appear as a
      three or four element list, starting with the symbol `status'; response messages appear
      as a list, starting with the symbol `reply' and `self' messages appear as a two element
      list, starting with the symbol `self'.}

  \objListIOItem{bang}{an error was detected}
  
  \objListIOEnd

\objItemCompanion{none}

\objItemStandalone{no, works with another \objName{udpPort} object on the same computer or another
computer that is reachable via a TCP/IP network}

\objItemRetainsState{yes}

\objItemCompatibility{\MaxName{} 4.x \{OS 9 and OS X\}}

\objItemFat{PPC-only}

\objItemCommands[]

  \objListCmdBegin

  \objListCmdItem{\emphcorr{bang}}{}
   Report the state of the communication (`unbound', `bound' or `unknown') as a three element list, with the symbol `status' as its first element.
   The third element is either `raw' or `max'.

  \objListCmdItem{\emphcorr{float}}{}
  Send the given floating point value to the other \objNameX{udpPort} object.

  \objListCmdItem{\emphcorr{integer}}{}
  Send the given integer value to the other \objNameX{udpPort} object.

  \objListCmdItem{list}{anything}
  Send the given list to the other \objNameX{udpPort} object.

  \objListCmdItem{mode}{symbol}{}
  Set the operating mode of the communication to either `raw' or `max'.
  In `raw' mode, data is transferred without any structure; in `max' mode it is as described below.
  Both ends of the communication must agree on the mode.

  \objListCmdItem{port}{integer}
  Set the number of the port to communicate on.

  \objListCmdItem{self}{}
  Returns the IP address of this object as a two element list, with the symbol `self' as its first
  element.
  
  \objListCmdItem{send}{anything}
  Send the arguments to the other \objNameX{udpPortX} object.
  This message permits the transmission of an arbitrary list; in particular any list that could be
  confused with a command to the \objNameX{udpPort} object.


  \objListCmdItem{sendTo}{symbol integer}
  Specify the address of the machine running the other \objNameX{udpPort}
  object to communicate with, in `dotted' notation, along with the port to communicate on.
  
  \objListCmdItem{status}{}
   Report the state of the communication (`unbound', `bound' or `unknown') as a three element list, with the symbol `status' as its first element.
   The third element is either `raw' or `max'.

  \objListCmdItem{verbose}{\textnormal{[}on\textnormal{/}off\textnormal{]}}
  Communication tracing to the \MaxName{} window will be enabled (`on'), disabled (`off') or
  reversed, if no argument is given.
  
  \objListCmdEnd

\objItemFile

\objItemMessage[see \objName{tcpClient}]

\objItemComments[Once a communication path is established between a \objName{udpPort} object and
another \objNameX{udpPort} object, either object can send messages---the path is full-duplex.
Note as well that only there can only be one \objName{tcpMultiServer}, \objName{tcpServer} or \objNameX{udpPort} object for any given port.]{}

\objEnd{\objNameE{udpPort}}
