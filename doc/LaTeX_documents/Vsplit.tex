\ProvidesFile{Vsplit.tex}[v1.0.0]
\startObject{\objNameS{Vsplit}}{Vsplit}
\index{Themes!Vector~manipulation!Vsplit}
\index{Vectors!Dyadic~operations!Vsplit}
\objPicture{Vsplitsymbol.ps}
\objItemDescription{\objNameD{Vsplit} is a combination of a \objName{Vtake} and \objName{Vdrop}.}

\objItemCreated{July 2005}

\objItemVersion{1.0.0}

\objItemHelp{yes}

\objItemTheme{Vector manipulation}

\objItemClass{Lists}

\objItemArgs{\nothing}

  \objListArgBegin
  \objListArgItem{how-many}{integer}{the number of elements to split by.
     A positive number indicates that the split point is counted from the beginning of the input list,
     while a negative number indicates that the split point is counted from the end of the list.}
  \objListArgEnd
  
\objItemInlet{\nothing}

  \objListIOBegin
  \objListIOItem{bang\textnormal{/}list}{the list to be split}

  \objListIOItem{integer}{the number of elements to split.
  This replaces the initial argument.}
  
  \objListIOEnd

\objItemOutlet{\nothing}

  \objListIOBegin
  \objListIOItem{list}{the left part of the list, or the previous result (if a `bang' is received)}
  \objListIOEnd

  \objListIOBegin
  \objListIOItem{list}{the right part of the list, or the previous result (if a `bang' is received)}
  \objListIOEnd

\objItemCompanion{none}

\objItemStandalone{yes}

\objItemRetainsState{yes, the number of elements to split}

\objItemCompatibility{\MaxName{} 4.x \{OS 9 and OS X\}}

\objItemFat{PPC-only}

\objItemCommands[]

  \objListCmdBegin
  \objListCmdItem{\emphcorr{bang}}{}
  Return the previous result, if any.
  \objListCmdEnd

\objItemFile

\objItemMessage

\objItemComments

\objEnd{\objNameE{Vsplit}}

% $Log: Vsplit.tex,v $
% Revision 1.4  2006/07/20 04:47:50  churchoflambda
% Re-added the files to record their changes.
%
% Revision 1.2  2005/08/02 15:07:04  churchoflambda
% Added CVS tags; add rail diagrams for pfsm, map1d, map2d, map3d and listen.
%
