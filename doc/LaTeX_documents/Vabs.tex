\ProvidesFile{Vabs.tex}[v1.0.2]
\startObject{\objNameS{Vabs}}{Vabs}
\index{Themes!Miscellaneous!Vabs}
\index{Themes!Vector~manipulation!Vabs}
\index{Vectors!Monadic~operations!Vabs}
\objPicture{Vabssymbol.ps}
\objItemDescription{\objNameD{Vabs} calculates the absolute value of the input
(either a list or a single number).}

\objItemCreated{April 2001}

\objItemVersion{1.0.2}

\objItemHelp{yes}

\objItemTheme{Miscellaneous}

\objItemClass{Arith/Logic/Bitwise, Lists}

\objItemArgs{none}

\objItemInlet{\ }

  \objListIOBegin
  \objListIOItem{anything}{the input to be processed}
  \objListIOEnd

\objItemOutlet{\ }

  \objListIOBegin
  \objListIOItem{anything}{the input after calculating the absolute value, or the previous result
      (if a `bang' is received)}
  \objListIOEnd

\objItemCompanion{none}

\objItemStandalone{yes}

\objItemRetainsState{no}

\objItemCompatibility{\MaxName{} 3.x and \MaxName{} 4.x \{OS 9 and OS X\}}

\objItemFat{Fat}

\objItemCommands[]

  \objListCmdBegin
  \objListCmdItem{\emph{bang}}{}
  Return the previous result, if any.
  \objListCmdEnd

\objItemFile

\objItemMessage

\objItemComments[In mathematical terms: $y_i \gets | x_i |$, where $x = x_1,x_2,\dots,x_n$ is the inlet value and
$y = y_1,y_2,\dots,y_n$ is the outlet result.]

\objEnd{\objNameE{Vabs}}
