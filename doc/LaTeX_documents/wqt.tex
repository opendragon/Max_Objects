\ProvidesFile{wqt.tex}[v1.0.4]
\startObject{\objNameS{wqt}}{wqt}
\index{Themes!QuickTime\texttrademark!wqt}
\objPicture{wqtsymbol.ps}
\objItemDescription{\objNameD{wqt} provides a windowed interface to QuickTime\texttrademark{} movies,
permitting control of playback rate and the section of the movie to be played.
It provides more functionality than the standard \objNameS{movie} object.}

\objItemCreated{October 1998}

\objItemVersion{1.0.4}

\objItemHelp{yes}

\objItemTheme{QuickTime\texttrademark}

\objItemClass{Graphics}

\objItemArgs{\nothing}

  \objListArgBegin
  \objListArgItem{init-movie}{(optional) symbol}{the initial movie to be loaded}
  \objListArgEnd

\objItemInlet{\nothing}

  \objListIOBegin
  \objListIOItem{list}{the command input}
  \objListIOEnd

\objItemOutlet{\nothing}

  \objListIOBegin
  \objListIOItem{integer}{the result of a command}

  \objListIOItem{integer}{the duration of the current movie}

  \objListIOItem{bang}{playing has stopped}

  \objListIOItem{bang}{an error was detected}
  
  \objListIOEnd

\objItemCompanion{yes, (optional) the standard \objNameS{movie} play controller interface object can be
  attached to a \objNameX{wqt} object}

\objItemStandalone{yes}

\objItemRetainsState{yes}

\objItemCompatibility{\MaxName{} 3.x and \MaxName{} 4.x \{OS 9 and OS X\}}

\objItemFat{Fat}

\objItemCommands[]

  \objListCmdBegin

  \objListCmdItem{active}{\textnormal{[}0\textnormal{/}1\textnormal{]}}
  Set the current movie active (`1') or inactive (`0').

  \objListCmdItem{\emphcorr{bang}}{}
  Start the current movie playing.

  \objListCmdItem{begin}{}
  Move to the beginning of the current movie and make it active.

  \objListCmdItem{count}{}
  Return the number of movies loaded.

  \objListCmdItem{duration}{}
  Return the length of the current movie.

  \objListCmdItem{end}{}
  Move to the end of the current movie and make it active.

  \objListCmdItem{getrate}{}
  Return the rate at which the current movie will be played.

  \objListCmdItem{getvolume}{}
  Return the audio level for the current movie.

  \objListCmdItem{\emphcorr{integer}}{}
  Move to the given frame number in the current movie.

  \objListCmdItem{load}{\textnormal{[}movie-name\textnormal{]}}
  Add the specified movie to the list of movies and make it the current movie.
  \objCmdArg{movie-name} must be a symbol, not a number.

  \objListCmdItem{mute}{\textnormal{[}0\textnormal{/}1\textnormal{]}}
  Change the audio level of the current movie, silencing it (`0') or restoring the previous level
  (`1').

  \objListCmdItem{pause}{}
  Stop the current movie.

  \objListCmdItem{rate}{integer \textnormal{[}integer\textnormal{]}}
  Set the rate at which the current movie will be played, using the ratio of the first number to
  the second, and start the movie playing.
  If only one number is given, or the second number is zero, assume that the second number has
  the value one.

  \objListCmdItem{resume}{}
  Continue playing the current movie after a \objCmdQ{pause} or a \objCmdQ{stop} command.

  \objListCmdItem{segment}{integer \textnormal{[}integer\textnormal{]}}
  Set the portion of the current movie that will be played to the section from the first frame
  number to the second.
  If the first number is zero and the second number is zero or less, set the portion to be the
  whole movie.
  If the second number is negative, set the portion to be from the first frame number to the end
  of the movie.
  That is, `0 0' is the whole movie, as is `0 $-1$', while `15 $-1$' is the portion from frame 15
  to the end.

  \objListCmdItem{start}{}
  Move to the beginning of the current movie, make it active and start it playing.

  \objListCmdItem{stop}{}
  Stop the current movie.

  \objListCmdItem{time}{}
  Return the current frame number of the current movie.

  \objListCmdItem{unload}{\textnormal{[}movie-name\textnormal{]}}
  If no movie is specified, remove the current movie from the list of movies.
  Otherwise, remove the specified movie from the list of movies.
  \objCmdArg{movie-name} must be a symbol, not a number.

  \objListCmdItem{volume}{\textnormal{[}integer\textnormal{]}}
  Set the audio level of the current movie.
  The maximum level is 255; setting the level negative acts to mute the current movie, but
  the \objCmdQ{mute} command can restore the audio level to the corresponding positive value.
  
  \objListCmdEnd

\objItemFile[QuickTime\texttrademark{} movie]

\objItemMessage

\objItemComments[The \objNameX{wqt} object was designed to address a critical weakness of the standard
\objNameS{movie} object: there was no way to request only a section of the movie be played, even
though QuickTime\texttrademark{} supports this ability.
One feature of the standard \objNameS{movie} object was not retained---mouse motion over the
\objNameX{wqt} object is not detected.]

\objEnd{\objNameE{wqt}}
