\ProvidesFile{dataType.tex}[v1.0.3]
\startObject{\objNameS{dataType}}{dataType}
\index{Themes!Miscellaneous!dataType}
\index{Themes!Programming~aids!dataType}
\objPicture{dataTypesymbol.ps}
\objItemDescription{\objNameD{dataType} returns a numeric code corresponding to the value
it receives.}

\objItemCreated{October 1998}

\objItemVersion{1.0.3}

\objItemHelp{yes}

\objItemTheme{Miscellaneous}

\objItemClass{Lists}

\objItemArgs{none}

\objItemInlet{\nothing}

  \objListIOBegin
  \objListIOItem{anything}{any value}
  \objListIOEnd

\objItemOutlet{\nothing}

  \objListIOBegin
  \objListIOItem{integer}{the code for the input (unknown~=~0, bang~=~1, float~=~2,
       integer~=~3, list~=~4, symbol~=~5)}
  \objListIOEnd

\objItemCompanion{none}

\objItemStandalone{yes}

\objItemRetainsState{no}

\objItemCompatibility{\MaxName{} 3.x and \MaxName{} 4.x \{OS 9 and OS X\}}

\objItemFat{Fat}

\objItemCommands

\objItemFile

\objItemMessage

\objItemComments

\objEnd{\objNameE{dataType}}
