\ProvidesFile{phidg_rfid.tex}[v1.0.1]
\startPlugin{\pluginNameS{Phidgets}{RFID}}{Phidgets}{RFID}

\pluginItemDescription{\pluginNameD{Phidgets}{RFID} corresponds to the PhidgetRFID device,
which reads RFID (\emphcorr{R}adio \emphcorr{F}requency \emphcorr{ID}entification) tags.
Note that the RFID plugin will generate messages asynchronously,
when an RFID tag is brought in to close proximity of the device.}

\pluginItemCreated{December 2003}

\pluginItemVersion{1.0.1}

\pluginItemCommands[]

  \pluginListCmdBegin

  \pluginListCmdItem{standard-do}{on\textnormal{/}off}
  Asynchronous notification of RFID tags will be enabled (`on') or disabled (`off').
  If asynchronous notification is enabled,
  when a tag is placed close to the RFID device,
  the list `\pluginCmd{\textitcorr{standard-response}} RFIDtag' will be sent out the report output of the \objName{fidget} object.
  The value `RFIDtag' is a symbol composed from the hexadecimal representation of the RFID tag,
  prefixed with an underscore character (``\_'').
  Note that the list is only sent when the tag approaches the device,
  and a new list is output when a different tag appears.
  
  \pluginListCmdItem{standard-get}{}
  Sends the current RFID tag out the report output of the \objName{fidget} object,
  as the list `\pluginCmd{\textitcorr{standard-response}} RFIDtag'.
  The value `RFIDtag' is a symbol composed from the hexadecimal representation of the RFID tag,
  prefixed with an underscore character (``\_'').

  \pluginListCmdEnd

\pluginItemComments[Note that \pluginNameS{Phidgets}{RFID} can be polled (using the \pluginCmdQ{get} command)
or configured to asynchronously report the presence of RFID tags (using the \pluginCmdQ{do} command).]

\pluginEnd{\pluginNameE{Phidgets}{RFID}}

% $Log: phidg_rfid.tex,v $
% Revision 1.5  2006/07/20 04:47:53  churchoflambda
% Re-added the files to record their changes.
%
% Revision 1.3  2005/08/02 15:07:09  churchoflambda
% Added CVS tags; add rail diagrams for pfsm, map1d, map2d, map3d and listen.
%
