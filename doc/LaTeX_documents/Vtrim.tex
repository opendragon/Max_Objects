\ProvidesFile{Vtrim.tex}[v1.0.0]
\startObject{\objNameS{Vtrim}}{Vtrim}
\index{Themes!Vector~manipulation!Vtrim}
\objPicture{Vtrimsymbol.ps}
\objItemDescription{\objNameD{Vtrim} is used to remove `noise' numbers from the beginning and end of a list.}

\objItemCreated{June 2003}

\objItemVersion{1.0.0}

\objItemHelp{yes}

\objItemTheme{Vector manipulation}

\objItemClass{Arith/Logic/Bitwise, Lists}

\objItemArgs{\nothing}

  \objListArgBegin
  \objListArgItem{separator1}{integer}{a `noise' number to remove.
    Only non-zero numbers will be recognized.}
  \objListArgItem{separator2}{(optional) integer}{another `noise' number to remove.}
  \objListArgItem{separator3}{(optional) integer}{another `noise' number to remove.}
  \objListArgItem{separator4}{(optional) integer}{another `noise' number to remove.}
  \objListArgItem{separator5}{(optional) integer}{another `noise' number to remove.} 
  \objListArgEnd

\objItemInlet{\nothing}

  \objListIOBegin
  \objListIOItem{integer\textnormal{/}list\textnormal{/}bang}{the list to be processed.
     A single number is treated as a single element list.}
  \objListIOEnd

\objItemOutlet{\nothing}

  \objListIOBegin
  \objListIOItem{list}{the reduced list}

  \objListIOItem{bang}{an empty list was generated}
  
  \objListIOEnd

\objItemCompanion{none}

\objItemStandalone{yes}

\objItemRetainsState{yes, the separator numbers}

\objItemCompatibility{\MaxName{} 4.x \{OS 9 and OS X\}}

\objItemFat{PPC-only}

\objItemCommands[]

  \objListCmdBegin
  \objListCmdItem{\emphcorr{bang}}{}
  Return the previous result, if any.
  \objListCmdEnd

\objItemFile

\objItemMessage

\objItemComments

\objEnd{\objNameE{Vtrim}}

% $Log: Vtrim.tex,v $
% Revision 1.5  2006/07/20 04:47:50  churchoflambda
% Re-added the files to record their changes.
%
% Revision 1.3  2005/08/02 15:07:08  churchoflambda
% Added CVS tags; add rail diagrams for pfsm, map1d, map2d, map3d and listen.
%
