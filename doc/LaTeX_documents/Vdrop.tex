\ProvidesFile{Vdrop.tex}[v1.0.4]
\startObject{\objNameS{Vdrop}}{Vdrop}
\index{Themes!Vector~manipulation!Vdrop}
\index{Vectors!Dyadic~operations!Vdrop}
\objPicture{Vdropsymbol.ps}
\objItemDescription{\objNameD{Vdrop} is an implementation of the \compLang{APL} `drop' operator
($\downarrow$)\index{APL!drop}, which is used to return the remainder of a vector (in \MaxName, a list) with
leading or trailing elements removed.}

\objItemCreated{July 2000}

\objItemVersion{1.0.4}

\objItemHelp{yes}

\objItemTheme{Vector manipulation}

\objItemClass{Lists}

\objItemArgs{\ }

  \objListArgBegin
  \objListArgItem{how-many}{integer}{the number of elements to drop.
     A positive number indicates that the elements are removed from the beginning of the input list,
     while a negative number indicates that the elements are to be removed from the end of the list.}
  \objListArgEnd
  
\objItemInlet{\ }

  \objListIOBegin
  \objListIOItem{bang\textnormal{/}list}{the list to be reduced}

  \objListIOItem{integer}{the number of elements to drop.
  This replaces the initial argument.}
  
  \objListIOEnd

\objItemOutlet{\ }

  \objListIOBegin
  \objListIOItem{list}{the reduced list, or the previous result (if a `bang' is received)}
  \objListIOEnd

\objItemCompanion{none}

\objItemStandalone{yes}

\objItemRetainsState{yes, the number of elements to drop}

\objItemCompatibility{\MaxName{} 3.x and \MaxName{} 4.x \{OS 9 and OS X\}}

\objItemFat{Fat}

\objItemCommands[]

  \objListCmdBegin
  \objListCmdItem{\emph{bang}}{}
  Return the previous result, if any.
  \objListCmdEnd

\objItemFile

\objItemMessage

\objItemComments

\objEnd{\objNameE{Vdrop}}
