\ProvidesFile{gvp100.tex}[v1.0.5]
\startObject{\objNameS{gvp100}}{gvp100}
\index{Themes!Device~interface!gvp100}
\objPicture{gvp100symbol.ps}
\objItemDescription{\objNameD{gvp100} is an interface to an ECHOlab DV7 video switcher,
which utilizes the Grass Valley Protocol.
It sends commands to a \objNameS{serial} or a \objReference{serialX} object,
which controls the serial port that the video switcher is attached to,
and responds to data returned from the video switcher via the \objNameS{serial} or \objName{serialX}
object.}

\objItemCreated{June 1998}

\objItemVersion{1.0.5}

\objItemHelp{no}

\objItemTheme{Device interface}

\objItemClass{Devices}

\objItemArgs{\nothing}

  \objListArgBegin
  \objListArgItem{address}{integer}{the select address of the video switcher.
       The address must be even and less than 256.}

  \objListArgItem{poll-rate}{(optional) integer}{the rate (in milliseconds) at which the companion
       \objNameS{serial} or \objName{serialX} object is polled via a sample request.
       The default rate is 100 milliseconds between sample requests.}

  \objListArgItem{size}{(optional) integer}{the size of the command pool that is used to buffer
       commands to the video switcher.
       The default size is 50 and the maximum size is 200.}

  \objListArgEnd

\objItemInlet{\nothing}

  \objListIOBegin
  \objListIOItem{list}{the command channel}

  \objListIOItem{anything}{the output of the companion \objNameS{serial} or \objName{serialX} object}
  \objListIOEnd

\objItemOutlet{\nothing}

  \objListIOBegin
  \objListIOItem{bang}{the sequence has completed}

  \objListIOItem{bang}{the command has completed}

  \objListIOItem{bang}{the sample request to send to the companion \objNameS{serial} or \objName{serialX} object}

  \objListIOItem{anything}{the data to send to the companion \objNameS{serial} or \objName{serialX} object}

  \objListIOItem{bang}{the \objCmdQ{break} request to send to the companion \objNameS{serial} or \objName{serialX} object,
       via a message object}

  \objListIOItem{bang}{an error was detected}

  \objListIOEnd

\objItemCompanion{works with \objName{serialX} objects with OS 9, or \objNameS{serial} objects with OS X}

\objItemStandalone{yes}

\objItemRetainsState{yes}

\objItemCompatibility{\MaxName{} 3.x and \MaxName{} 4.x \{OS 9 and OS X\}}

\objItemFat{Fat}

\objItemCommands[]

  \objListCmdBegin
  
  \objListCmdItem{allstop}{}
  Send the `allstop' command to the video switcher.

  \objListCmdItem{breakdone}{}
  The \objCmdQ{break} sent to the \objNameS{serial} or \objName{serialX} object has completed.
  This command shouldn't be sent under other circumstances.

  \objListCmdItem{c}{bus-settings}
  This is a shortcut for the \objCmdQ{crosspoint \objCmdArg{bus-settings}} command.

  \objListCmdItem{crosspoint}{bus-settings}
  The argument \objCmdArg{bus-settings} is a series of pairs of symbols and symbols or integers---
  if the list is of odd length, the value zero is added at the end.
  The first value is the name of the bus of the video switcher that is to be set (`pgm'/`program',
  `preset'/`preview', `key'/`insert'/`k'/`i') and the second value is the source for the given bus
  (`black', `background'/`b' or an integer between 0 and 9, where `black' is the same as `0' and
  `background' is the same as `9').
  Each of the resulting pairs is sent to the video switcher to change the program, preset or key busses.

  \objListCmdItem{!c}{level}
  This is a shortcut for the \objCmdQ{!clip \objCmdArg{level}} command.

  \objListCmdItem{!clip}{level}
  Set the clipping level to the given value, \objCmdArg{level}, which is a floating point number
  that is between 0 and 100.
  The command is the equivalent to \objCmdQ{dskanalogcontrol 8 \objCmdArg{level}}.

  \objListCmdItem{d}{control-values}
  A shortcut for the \objCmdQ{dskanalogcontrol \objCmdArg{control-values}} command.

  \objListCmdItem{dskanalogcontrol}{control-values}
  The argument \objCmdArg{control-values} is a series of pairs of floating point numbers---if the
  list is of odd length, the value zero is added at the end.
  The first number in each pair is the control to be set and the second number in each pair is the
  value to set the control to.
  Each of the resulting pairs is sent to the video switcher to change the DSK analog control settings.

  \objListCmdItem{e}{control-values}
  This is a shortcut for the \objCmdQ{effectsanalogcontrol \objCmdArg{control-values}} command.

  \objListCmdItem{effectsanalogcontrol}{control-values}
  The argument \objCmdArg{control-values} is a series of pairs of floating point numbers---if the
  list is of odd length, the value zero is added at the end.
  The first number in each pair is the control to be set and the second number in each pair is the
  value to set the control to.
  Each of the resulting pairs is sent to the video switcher to change the effects analog control
  settings.

  \objListCmdItem{!e}{position}
  This is a shortcut for the \objCmdQ{!effects \objCmdArg{position}} command.

  \objListCmdItem{!effects}{position}
  Set the `effects' control position of the video switcher to the given value, \objCmdArg{position},
  which is a floating point value between 0 and 100.
  The command is equivalent to \objCmdQ{effectsanalogcontrol 21 \objCmdArg{position}}.

  \objListCmdItem{endsequence}{}
  Add the pseudo-command `endsequence' to the buffer of commands to be sent to the video switcher.
  The pseudo-command `endsequence' is not actually sent to the video switcher, but results in a
  \objCmdQ{bang} being sent out the first outlet to signal that a sequence of commands has been
  completed.

  \objListCmdItem{!j}{horizontal vertical}
  This is a shortcut for the \objCmdQ{!joystick \objCmdArg{horizontal} \objCmdArg{vertical}} command.

  \objListCmdItem{!joystick}{horizontal vertical}
  Set the coordinates of the joystick on the video switcher.
  The given values, \objCmdArg{horizontal} and \objCmdArg{vertical}, are floating point numbers
  between 0 and 100.
  The command is equivalent to \objCmdQ{effectsanalogcontrol 18 \objCmdArg{horizontal}
  17 \objCmdArg{vertical}}.

  \objListCmdItem{learn-emem}{\textnormal{[}register\textnormal{]}}
  The video switcher settings will be stored in the given \objCmdArg{register} of the video switcher,
  where \objCmdArg{register} is an integer between 0 and 15.
  If no value for \objCmdArg{register} is given, zero is assumed.

  \objListCmdItem{m}{key\-Or\-Back\-ground1 \textnormal{[}key\-Or\-Back\-ground2\textnormal{]}}
  This is a shortcut for the \objCmdQ{transitionmode \objCmdArg{key\-Or\-Back\-ground1}
  [\objCmdArg{key\-Or\-Back\-ground2}]} command.

  \objListCmdItem{off}{buttons}
  The argument \objCmdArg{buttons} is a list of numbers, which are interpreted as the indices of the
  buttons and switches of the video switcher.
  The indices are between 0 and 80, but not all values correspond to actual buttons or switches.
  Each of the matching buttons and switches on the video switcher will have their state set to `off'.
  If an erroneous index appears, the remaining indices will be ignored.

  \objListCmdItem{on}{buttons}
  The argument \objCmdArg{buttons} is a list of numbers, which are interpreted as the indices of the
  buttons and switches of the video switcher.
  The indices are between 0 and 80, but not all values correspond to actual buttons or switches.
  Each of the matching buttons and switches on the video switcher will have their state set to `on'.
  If an erroneous index appears, the remaining indices will be ignored.

  \objListCmdItem{p}{buttons}
  This is a shortcut for the \objCmdQ{push \objCmdArg{buttons}} command.

  \objListCmdItem{push}{buttons}
  The argument \objCmdArg{buttons} is a list of numbers, which are interpreted as the indices of the
  buttons and switches of the video switcher.
  The indices are between 0 and 80, but not all values correspond to actual buttons or switches.
  Each of the matching buttons and switches on the video switcher will have their state flipped
  (from `off' to `on' or vice-versa).
  If an erroneous index appears, the remaining indices will be ignored.

  \objListCmdItem{r}{transition \textnormal{[}rate \textnormal{[}key\-Or\-Back\-ground\-Or\-Do\-It1
  \textnormal{[}key\-Or\-Back\-ground\-Or\-Do\-It2
  \textnormal{[}key\-Or\-Back\-ground\-Or\-Do\-It3\textnormal{]]]]}}
  This is a shortcut for the \objCmdQ{transitionrate \objCmdArg{transition} [\objCmdArg{rate}
  [\objCmdArg{key\-Or\-Back\-ground\-Or\-Do\-It1} [\objCmdArg{key\-Or\-Back\-ground\-Or\-Do\-It2}
  [\objCmdArg{key\-Or\-Back\-ground\-Or\-Do\-It3}]]]]} command.

  \objListCmdItem{recall-emem}{\textnormal{[}register\textnormal{]}}
  The previously stored video switcher settings will be retrieved from the given \objCmdArg{register}
  of the video switcher, where \objCmdArg{register} is an integer between 0 and 15.
  If no value for \objCmdArg{register} is given, zero is assumed.

  \objListCmdItem{!t}{position}
  This is a shortcut for the \objCmdQ{!take \objCmdArg{position}} command.

  \objListCmdItem{!take}{position}
  Set the `take' control position of the video switcher to the given value, \objCmdArg{position},
  which is a floating point value between 0 and 100.
  The command is equivalent to \objCmdQ{effectsanalogcontrol 11 \objCmdArg{position}}.

  \objListCmdItem{transitionmode}{key\-Or\-Back\-ground1
  \textnormal{[}key\-Or\-Back\-ground2\textnormal{]}}
  Set the active transition mode of the video switcher to either key (`key' or `k') or background
  (`background' or `b') or both.

  \objListCmdItem{transitionrate}{transition \textnormal{[}rate
  \textnormal{[}key\-Or\-Back\-ground\-Or\-Do\-It1
  \textnormal{[}key\-Or\-Back\-ground\-Or\-Do\-It2
  \textnormal{[}key\-Or\-Back\-ground\-Or\-Do\-It3\textnormal{]]]]}}
  Set the transition rate for the given \objCmdArg{transition} (`auto'/`a', `dsk'/`d' or `f2b'/`f',
  where `f2b' represents fade-to-black) to the given value, \objCmdArg{rate},
  which is an integer between 0 and 1000.
  The transition rate affects either the key (`key' or `k') or background (`background' or `b')
  transitions.
  If the symbol `doit' or `d' appears, the transition is triggered as well.
  If no value for \objCmdArg{rate} is given, zero is assumed.

  \objListCmdItem{v}{\textnormal{[}on\textnormal{/}off\textnormal{]}}
  This is a shortcut for the \objCmdQ{verbose [\objCmdArg{on}/\objCmdArg{off}]} command.

  \objListCmdItem{verbose}{\textnormal{[}on\textnormal{/}off\textnormal{]}}
  Communication tracing to the \MaxName{} window will be enabled (`on'), disabled (`off') or reversed,
  if no argument is given.

  \objListCmdItem{w}{\textnormal{[}index\textnormal{]}}
  This is a shortcut for the \objCmdQ{wipepattern [\objCmdArg{index}]} command.

  \objListCmdItem{wipepattern}{\textnormal{[}index\textnormal{]}}
  Set the wipe pattern of the video switcher to \objCmdArg{index}, which is one of the following
  integer values: 0, 1, 3, 4, 10, 11, 20, 23, 30 or 33.
  If no value for \objCmdArg{index} is given, zero is assumed.

  \objListCmdItem{x}{}
  This is a shortcut for the \objCmdQ{xreset} command.

  \objListCmdItem{xreset}{}
  Remove any pending commands for the video switcher and send a `break' to the video switcher to
  force a device reset.
  
  \objListCmdEnd

\objItemFile

\objItemMessage

\objItemComments[Figure~\objImageReference{diagram:gvp100connect1} shows how to connect a
\objNameX{gvp100} object to a \objNameS{serial} object.
Figure~\objImageReference{diagram:gvp100connect2} shows how to connect a
\objNameX{gvp100} object to a \objName{serialX} object.]
\objDiagram{gvp100connections1.ps}{gvp100connect1}{OS X: Connecting a \objNameX{gvp100} object to a
\objNameS{serial} object}
\objDiagram{gvp100connections2.ps}{gvp100connect2}{OS 9: Connecting a \objNameX{gvp100} object to a
\objName{serialX} object}

\objEnd{\objNameE{gvp100}}

% $Log: gvp100.tex,v $
% Revision 1.4  2006/03/25 21:51:18  churchoflambda
% Added the 'senseX' object and modified the connection diagrams to show 'serial' as well as 'serialX'.
%
% Revision 1.3  2005/08/02 15:07:09  churchoflambda
% Added CVS tags; add rail diagrams for pfsm, map1d, map2d, map3d and listen.
%

