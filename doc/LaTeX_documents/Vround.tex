\ProvidesFile{Vround.tex}[v1.0.5]
\startObject{\objNameS{Vround}}{Vround}
\index{Themes!Miscellaneous!Vround}
\index{Themes!Vector~manipulation!Vround}
\index{Vectors!Monadic~operations!Vround}
\objPicture{Vroundsymbol.ps}
\objItemDescription{\objNameD{Vround} calculates the integer nearest to the value given (either a
list or a single number).}

\objItemCreated{November 2000}

\objItemVersion{1.0.5}

\objItemHelp{yes}

\objItemTheme{Miscellaneous}

\objItemClass{Arith/Logic/Bitwise, Lists}

\objItemArgs{\nothing}

  \objListArgBegin
  \objListArgItem{mode}{(optional) symbol}{either `f', `i' or `m' to indicate whether the output
        is to be floating-point values only, integer values only, or mixed values.
        Mixed values are floating-point if the input is floating-point and integer if the input is
        integer.
        The default is `m'.}
  
  \objListArgEnd

\objItemInlet{\nothing}

  \objListIOBegin
  \objListIOItem{anything}{the input to be processed}
  \objListIOEnd

\objItemOutlet{\nothing}

  \objListIOBegin
  \objListIOItem{anything}{the input after calculating the nearest integer, or the previous result
      (if a `bang' is received)}
  \objListIOEnd

\objItemCompanion{none}

\objItemStandalone{yes}

\objItemRetainsState{no}

\objItemCompatibility{\MaxName{} 3.x and \MaxName{} 4.x \{OS 9 and OS X\}}

\objItemFat{Fat}

\objItemCommands

\objItemFile

\objItemMessage

\objItemComments

\objEnd{\objNameE{Vround}}
