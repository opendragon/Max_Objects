\ProvidesFile{queue.tex}[v1.0.2]
\startObject{\objNameS{queue}}{queue}
\index{Themes!Programming~aids!queue}
\objPicture{queuesymbol.ps}
\objItemDescription{\objNameD{queue} is an implementation of first-in-first-out queues.
Unlike conventional queues, the \objNameX{queue} object can store lists as well as single
values on each level.}

\objItemCreated{April 2001}

\objItemVersion{1.0.2}

\objItemHelp{yes}

\objItemTheme{Programming aids}

\objItemClass{Data}

\objItemArgs{\ }

  \objListArgBegin
  \objListArgItem{tag-name}{integer}{the maximum number of elements that can be held in the queue.
  This is also known as its depth.
  A depth of zero is interpreted as a queue that can hold an unlimited number of elements.}
  \objListArgEnd

\objItemInlet{\ }

  \objListIOBegin
  \objListIOItem{list}{the command input}
  \objListIOEnd

\objItemOutlet{\ }

  \objListIOBegin
  \objListIOItem{anything}{the retrieved data}

  \objListIOItem{integer}{the depth of the queue (the number of elements held in the
        queue)}

  \objListIOItem{bang}{an error was detected}
  
  \objListIOEnd

\objItemCompanion{none}

\objItemStandalone{yes}

\objItemRetainsState{yes}

\objItemCompatibility{\MaxName{} 3.x and \MaxName{} 4.x \{OS 9 and OS X\}}

\objItemFat{Fat}

\objItemCommands[]

  \objListCmdBegin
  
  \objListCmdItem{add}{anything}
  Place the given data at the end of the queue.
  If the queue was already full, output the first element in the queue (the oldest) before adding
  the new data.

  \objListCmdItem{\emph{bang}}{}
  Output all data held in the queue in the order in which it was received.

  \objListCmdItem{clear}{}
  Remove all data from the queue.

  \objListCmdItem{depth}{}
  Return the depth of the queue.
  
  \objListCmdItem{fetch}{}
  Return the first element of the queue, without removing it.

  \objListCmdItem{pull}{}
  Return the first element of the queue and remove it from the queue.

  \objListCmdItem{setdepth}{new-depth}
  Set the maximum depth of the queue to new-depth.
  If the new maximum depth is zero, the contents of the queue are left alone.
  If the new maximum depth is less than the current depth, all the data is removed from the queue.
  
  \objListCmdItem{trace}{\textnormal{[}on\textnormal{/}off\textnormal{]}}
  Queue update tracing to the \MaxName{} window will be enabled (`on'), disabled (`off')
      or reversed, if no argument is given.
  
  \objListCmdEnd

\objItemFile

\objItemMessage

\objItemComments

\objEnd{\objNameE{queue}}
