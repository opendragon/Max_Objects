\ProvidesFile{Vfloor.tex}[v1.0.5]
\startObject{\objNameS{Vfloor}}{Vfloor}
\index{Themes!Miscellaneous!Vfloor}
\index{Themes!Vector~manipulation!Vfloor}
\index{Vectors!Monadic~operations!Vfloor}
\objPicture{Vfloorsymbol.ps}
\objItemDescription{\objNameD{Vfloor} calculates the largest integer less than the value given
(either a list or a single number).}

\objItemCreated{November 2000}

\objItemVersion{1.0.5}

\objItemHelp{yes}

\objItemTheme{Miscellaneous}

\objItemClass{Arith/Logic/Bitwise, Lists}

\objItemArgs{\nothing}

  \objListArgBegin
  \objListArgItem{mode}{(optional) symbol}{either `f', `i' or `m' to indicate whether the output
        is to be floating-point values only, integer values only, or mixed values.
        Mixed values are floating-point if the input is floating-point and integer if the input is
        integer.
        The default is `m'.}
  
  \objListArgEnd

\objItemInlet{\nothing}

  \objListIOBegin
  \objListIOItem{anything}{the input to be processed}
  \objListIOEnd

\objItemOutlet{\nothing}

  \objListIOBegin
  \objListIOItem{anything}{the input after calculating the floor, or the previous result
     (if a `bang' is received)}
  \objListIOEnd

\objItemCompanion{none}

\objItemStandalone{yes}

\objItemRetainsState{no}

\objItemCompatibility{\MaxName{} 3.x and \MaxName{} 4.x \{OS 9 and OS X\}}

\objItemFat{Fat}

\objItemCommands[]

  \objListCmdBegin
  \objListCmdItem{\emphcorr{bang}}{}
  Return the previous result, if any.
  \objListCmdEnd

\objItemFile

\objItemMessage

\objItemComments[In mathematical terms: $y_i \gets \lfloor x_i \rfloor$, where $x_1,x_2,\dots,x_n$ is the inlet value and
$y_1,y_2,\dots,y_n$ is the outlet result.]

\objEnd{\objNameE{Vfloor}}

% $Log: Vfloor.tex,v $
% Revision 1.5  2006/07/20 04:47:49  churchoflambda
% Re-added the files to record their changes.
%
% Revision 1.3  2005/08/02 15:07:03  churchoflambda
% Added CVS tags; add rail diagrams for pfsm, map1d, map2d, map3d and listen.
%
