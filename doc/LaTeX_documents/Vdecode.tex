\ProvidesFile{Vdecode.tex}[v1.0.0]
\startObject{\objNameS{Vdecode}}{Vdecode}
\index{Themes!Vector~manipulation!Vdecode}
\index{Vectors!Dyadic~operations!Vdecode}
\objPicture{Vdecodesymbol.ps}
\objItemDescription{\objNameD{Vdecode} is an implementation of the \compLang{APL} `decode' operator
($\bot$)\index{APL!decode}, which is used to convert a coded representation of a number into the number itself.}

\objItemCreated{June 2003}

\objItemVersion{1.0.0}

\objItemHelp{yes}

\objItemTheme{Vector manipulation}

\objItemClass{Arith/Logic/Bitwise, Lists}

\objItemArgs{\nothing}

  \objListArgBegin
  \objListArgItem{base1}{integer}{the leftmost base to be applied.
     If this is the only base it will be repeated as often as needed to do the conversion.
     If there are more bases and this base is negative, it will be repeated (as it's absolute value) as often as needed to do the conversion,
     after the other bases have been applied.
     Bases must be greater than 1.}
  \objListArgItem{base2}{(optional) integer}{the next-to-last base to apply.}
  \objListArgItem{base3}{(optional) integer}{the next-to-next-to-last base to apply.}
  \objListArgItem{base4}{(optional) integer}{the next-to-$\ldots$-last base to apply.}
  \objListArgItem{base5}{(optional) integer}{the last base to apply.
     The elements of the input list are multiplied by each base, in sequence, and the results are summed into the output number.} 
  \objListArgEnd

\objItemInlet{\nothing}

  \objListIOBegin
  \objListIOItem{list\textnormal{/}number}{the list to be decoded.
     A single number is treated as a single element list.}
  \objListIOEnd

\objItemOutlet{\nothing}

  \objListIOBegin
  \objListIOItem{number}{the decoded result}
  \objListIOEnd

\objItemCompanion{none}

\objItemStandalone{yes}

\objItemRetainsState{no}

\objItemCompatibility{\MaxName{} 4.x \{OS 9 and OS X\}}

\objItemFat{PPC-only}

\objItemCommands

\objItemFile

\objItemMessage

\objItemComments

\objEnd{\objNameE{Vdecode}}
