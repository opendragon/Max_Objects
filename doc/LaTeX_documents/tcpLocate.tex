\ProvidesFile{tcpLocate.tex}[v1.0.2]
\startObject{\objNameS{tcpLocate}}{tcpLocate}
\index{Themes!TCP/IP!tcpLocate}
\objPicture{tcpLocatesymbol.ps}
\objItemDescription{\objNameD{tcpLocate} is an interface to the TCP/IP stack on a Macintosh,
providing a client to identify the IP address corresponding to an Internet address.}

\objItemCreated{November 2000}

\objItemVersion{1.0.2}

\objItemHelp{yes}

\objItemTheme{TCP/IP}

\objItemClass{Devices}

\objItemArgs{None}

\objItemInlet{\nothing}

  \objListIOBegin
  \objListIOItem{list}{the command input}
  \objListIOEnd

\objItemOutlet{\nothing}

  \objListIOBegin
  \objListIOItem{list}{the response as a symbol}

  \objListIOItem{bang}{an error was detected}
  
  \objListIOEnd

\objItemCompanion{none}

\objItemStandalone{no, works with a \objReference{tcpClient}, \objReference{tcpServer} or
   \objReference{tcpMultiServer} object}

\objItemRetainsState{yes}

\objItemCompatibility{\MaxName{} 3.x and \MaxName{} 4.x \{OS 9 and OS X\}}

\objItemFat{PPC-only}

\objItemCommands[]

  \objListCmdBegin

  \objListCmdItem{symbol}{}
  Interpret the given value as an Internet address and return the IP address if found.

  \objListCmdItem{verbose}{\textnormal{[}on\textnormal{/}off\textnormal{]}}
  Communication tracing to the \MaxName{} window will be enabled (`on'), disabled (`off') or
  reversed, if no argument is given.
  
  \objListCmdEnd

\objItemFile

\objItemMessage

\objItemComments

\objEnd{\objNameE{tcpLocate}}

% $Log: tcpLocate.tex,v $
% Revision 1.3  2005/08/02 15:07:10  churchoflambda
% Added CVS tags; add rail diagrams for pfsm, map1d, map2d, map3d and listen.
%
