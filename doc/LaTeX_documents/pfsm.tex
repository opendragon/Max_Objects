\ProvidesFile{pfsm.tex}[v1.0.4]
\startObject{\objNameS{pfsm}}{pfsm}
\index{Themes!Programming~aids!pfsm}
\objPicture{pfsmsymbol.ps}
\objItemDescription{\objNameD{pfsm} is an implementation of a Finite State Machine, with probabilistic transitions.
Hence, Probabilistic Finite State Machine, or PFSM.
It processes a sequence of messages against a state file, generating output as guided by the state transitions
triggered by the input.}

\objItemCreated{May 2000}

\objItemVersion{1.0.4}

\objItemHelp{yes}

\objItemTheme{Programming aids}

\objItemClass{Arith/Logic/Bitwise, Lists}

\objItemArgs{\nothing}

  \objListArgBegin
  \objListArgItem{init-file}{(optional) symbol}{the name of the state file to load initially}
  \objListArgEnd

\objItemInlet{\nothing}

  \objListIOBegin
  \objListIOItem{integer\textnormal{/}list}{the command input}
  \objListIOEnd

\objItemOutlet{\nothing}

  \objListIOBegin
  \objListIOItem{list}{the retrieved data}

  \objListIOItem{bang}{a stop state was reached}

  \objListIOItem{bang}{an error state was reached or an error was detected}
  
  \objListIOEnd

\objItemCompanion{none}

\objItemStandalone{yes}

\objItemRetainsState{yes}

\objItemCompatibility{\MaxName{} 3.x and \MaxName{} 4.x \{OS 9 and OS X\}}

\objItemFat{Fat}

\objItemCommands[]

  \objListCmdBegin
  
  \objListCmdItem{autorestart}{\textnormal{[}on\textnormal{/}off\textnormal{]}}
  Enable (`on') or disable (`off') autorestart.
  If autorestart is enabled, when the \objNameX{pfsm} object reaches a `stop' state it will
  automatically advance to the `start' state.
  In either case, a `bang' will always be sent to the stop state outlet.

  \objListCmdItem{clear}{}
  The current state will be cleared.
  This is not the same as going to the `start' state.

  \objListCmdItem{describe}{}
  The currently loaded set of transitions will be reported to the \MaxName{} window.

  \objListCmdItem{do}{list}
  The first element in the given list is matched against the transitions available for the
  current state.
  If a match is found, the current state is set to the destination of the transition and the
  output portion of the transition is processed using the remainder of the given list.

  \objListCmdItem{goto}{new-state}
  The current state will be set to \objCmdArg{new-state}, if it is in the currently loaded set
  of transitions.

  \objListCmdItem{\emphcorr{integer}}{}
  The given value is matched against the transitions available for the current state.
  If a match is found, the current state is set to the destination of the transition and the
  output portion of the transition is processed using an empty list as the remainder of the input.

  \objListCmdItem{list}{anything}
  The first element in the given list is matched against the transitions available for the
  current state.
  If a match is found, the current state is set to the destination of the transition and the
  output portion of the transition is processed using the remainder of the given list.

  \objListCmdItem{load}{filename}
  The currently loaded set of transitions will be set to the contents of the named state file.

  \objListCmdItem{start}{}
  The current state will be set to the start state for the currently loaded set of transitions
  and the \objNameX{pfsm} object will be set to `running'.

  \objListCmdItem{status}{}
  The state of the \objNameX{pfsm} object (running or stopped, the current state, the start state
  and whether autorestart is enabled) will be reported to the \MaxName{} window.

  \objListCmdItem{stop}{}
  The \objNameX{pfsm} object will be stopped and the current state will be cleared.

  \objListCmdItem{trace}{\textnormal{[}on\textnormal{/}off\textnormal{]}}
  State transition tracing to the \MaxName{} window will be enabled (`on'), disabled (`off') or
  reversed, if no argument is given.
  
  \objListCmdEnd

\objItemFile[]

\begin{quote}
The state file describes the legal transitions that the \objNameX{pfsm} object will perform and is
composed of five sections:
\begin{enumerate}[1)]
\item the state symbols---a list of symbols enclosed in square brackets (`[` and `]')
\item the starting state---a single symbol
\item the stop states---a list of symbols enclosed in square brackets
\item the error states---a list of symbols enclosed in square brackets
\item the transitions---a list of statements describing the transitions.
\end{enumerate}
The order of the transitions is critical---for each state, the first transition matching the
input provided (and meeting the probability criteria, if it is a probabilistic transition) will
be processed and the remaining transitions will be ignored.
For this reason, transitions with specific matching values should precede transitions with wild
card matching criteria.

Comments start with the `\#' character and end with the `;' character.

The transitions are in one of two forms, and are terminated with the `;' character:
\begin{enumerate}[5a)]
\item current-state input -$>$ new-state output
\item current-state input -? probability new-state output
\end{enumerate}
The output list can contain the symbol `\$' to indicate the matching input or the symbol `\$\$' to
indicate the overflow values (any values in the input list, after the matching element) or the
symbol `\$*' to indicate the new name of the new state.
The value `input' can be any of the following special symbols:
\begin{enumerate}[a)]
\item @s represents any valid symbol
\item @n represents any valid number
\item @r, followed by two numbers, represents a range of values
\item '?, where ? is any printable character, represents the numeric equivalent to the character,
and can appear wherever a number could appear
\item anything else represents the value that will be matched literally
\end{enumerate}
The probability is in the form of a fraction between 0.0 and 1.0
\end{quote}

\objItemMessage

\objItemComments[The \objNameX{pfsm} object was designed to address some problems that were found in
attempting to use \objNameS{table} objects to perform complex state sequencing.
I realized that a better approach was to represent the actions as the output of an FSM, and
added the probabilistic transition mechanism as a useful extension.
Figure~\objImageReference{diagram:pfsmstate} is a state
transition diagram for the example state file, Figure~\objImageReference{file:pfsmstate}.]

\objFileDescription[0.80]{An example state file for a \objNameX{pfsm} object}{pfsmstate}{
\#File: state-file;\\
\# Each line is terminated with a semicolon;\\
\# A comment starts with a '\#' character;\\
\# Note that white space is critical around symbols and operators;\\
\vspace{1ex}
\# State symbols;\\[0pt] % this is so that the next '[' is misinterpreted!
[ alpha omega beta test-state theta gamma ]\\
\vspace{1ex}
\# Starting state;\\
alpha\\
\vspace{1ex}
\# Stop states;\\[0pt] % this is so that the next '[' is misinterpreted!
[ theta ]\\
\vspace{1ex}
\# Error states;\\[0pt] % this is so that the next '[' is misinterpreted!
[ omega ]\\
\vspace{1ex}
\# Transitions;\\
\# The format is: current-state input -$>$ new-state output ;\\
\# or current-state input -? probability new-state output ;\\
\# where output can contain the symbol \$ to indicate the matching input;\\
\# the symbol \$\$ to indicate the overflow values or the symbol;\\
\# \$* to indicate the new state;\\
\# The special symbol @s represents any valid symbol on input;\\
\# The special symbol @n represents any valid number on input;\\
\# The special symbol @r represents a range of values (the next two numbers;\\
\#    in the input);\\
\# The special form '? represents the numeric equivalent to the character ?;\\
\#    for input;\\
\# The probability is in the form of a fraction between 0 and 1;\\
\vspace{1ex}
alpha 42 -$>$ beta \$ ;\\
\vspace{1ex}
alpha 'a -$>$ beta \$;\\
\vspace{1ex}
alpha @r 'b 'g -$>$ theta \$;\\
\vspace{1ex}
test garbage -$>$ test ;\\
\vspace{1ex}
beta blorg -$>$ alpha ;\\
\vspace{1ex}
beta blirg -$>$ theta \$ ;\\
\vspace{1ex}
alpha @s -$>$ gamma \$\$ \$;\\
\vspace{1ex}
gamma @n -$>$ beta chuck you \$ times;\\
\vspace{1ex}
theta @s -$>$ theta symbol;\\
\vspace{1ex}
theta @n -$>$ theta number;\\
\vspace{1ex}
gamma @s -? 0.30 beta whoo hoo; \#probabilistic transition to beta;\\
\vspace{1ex}
gamma @s -$>$ gamma; \#handle non-branch to beta;\\}

\objDiagram{pfsmstate.ps}{pfsmstate}{State transition diagram for the example state file}

\objEnd{\objNameE{pfsm}}
