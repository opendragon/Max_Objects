\ProvidesFile{Vrotate.tex}[v1.0.2]
\startObject{\objNameS{Vrotate}}{Vrotate}
\index{Themes!Vector~manipulation!Vrotate}
\index{Vectors!Dyadic~operations!Vrotate}
\objPicture{Vrotatesymbol.ps}
\objItemDescription{\objNameD{Vrotate} is an implementation of the \compLang{APL} `rotate' operator
(dyadic $\Phi$)\index{APL!rotate}, which is used to rotate the elements of a vector (in \MaxName, a list).}

\objItemCreated{April 2001}

\objItemVersion{1.0.2}

\objItemHelp{yes}

\objItemTheme{Vector manipulation}

\objItemClass{Lists}

\objItemArgs{\nothing}

  \objListArgBegin
  \objListArgItem{how-many}{integer}{the number of elements to rotate.
     A positive number indicates that the beginning of the input list is moved towards the end of th
     input list by the number of elements given,
     while a negative number indicates that the end of the input list is moved towards the beginning of the
     input list by (the absolute value of) the number of elements given.}
  \objListArgEnd
  
\objItemInlet{\nothing}

  \objListIOBegin
  \objListIOItem{bang\textnormal{/}list}{the list to be rotated}

  \objListIOItem{integer}{the number of elements to rotate.
  This replaces the initial argument.}
  
  \objListIOEnd

\objItemOutlet{\nothing}

  \objListIOBegin
  \objListIOItem{list}{the rotated list, or the previous result (if a `bang' is received)}
  \objListIOEnd

\objItemCompanion{none}

\objItemStandalone{yes}

\objItemRetainsState{yes, the number of elements to rotate}

\objItemCompatibility{\MaxName{} 3.x and \MaxName{} 4.x \{OS 9 and OS X\}}

\objItemFat{Fat}

\objItemCommands[]

  \objListCmdBegin
  \objListCmdItem{\emphcorr{bang}}{}
  Return the previous result, if any.
  \objListCmdEnd

\objItemFile

\objItemMessage

\objItemComments

\objEnd{\objNameE{Vrotate}}

% $Log: Vrotate.tex,v $
% Revision 1.3  2005/08/02 15:07:03  churchoflambda
% Added CVS tags; add rail diagrams for pfsm, map1d, map2d, map3d and listen.
%
