\ProvidesFile{notX.tex}[v1.0.4]
\startObject{\objNameS{notX}}{notX}
\index{Themes!Miscellaneous!notX}
\index{Vectors!Monadic~operations!notX}
\objPicture{notXsymbol.ps}
\objItemDescription{\objNameD{notX} provides the logical complement of a number or a list of numbers,
returning `0' for each non-zero number and `1' for each zero number.
Non-numeric values are returned without modification.}

\objItemCreated{September 1998}

\objItemVersion{1.0.4}

\objItemHelp{yes}

\objItemTheme{Miscellaneous}

\objItemClass{Arith/Logic/Bitwise, Lists}

\objItemArgs{none}

\objItemInlet{\nothing}

  \objListIOBegin
  \objListIOItem{anything}{the input to be processed}
  \objListIOEnd

\objItemOutlet{\nothing}

  \objListIOBegin
  \objListIOItem{anything}{the input after complementation, or the previous result (if a `bang' is
       received)}
  \objListIOEnd

\objItemCompanion{none}

\objItemStandalone{yes}

\objItemRetainsState{no}

\objItemCompatibility{\MaxName{} 3.x and \MaxName{} 4.x \{OS 9 and OS X\}}

\objItemFat{Fat}

\objItemCommands[]

  \objListCmdBegin
  \objListCmdItem{\emphcorr{bang}}{}
  Return the previous result, if any.
  \objListCmdEnd

\objItemFile

\objItemMessage

\objItemComments[The \objNameX{notX} object was created because there was a \objNameS{not}
object that wasn't Fat, when a Fat object was needed.
It was extended to handle lists and floating point values.]

\objEnd{\objNameE{notX}}
