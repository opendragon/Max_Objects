\ProvidesFile{Vscan.tex}[v1.0.2]
\startObject{\objNameS{Vscan}}{Vscan}
\index{Themes!Vector~manipulation!Vscan}
\index{Vectors!Monadic~operations!Vscan}
\objPicture{Vscansymbol.ps}
\objItemDescription{\objNameD{Vscan} is an implementation of the \compLang{APL} `scan' operator
($f\backslash$)\index{APL!scan}, which is used to apply an operator ($f$) over the elements of a vector
(in \MaxName, a list).}

\objItemCreated{November 2000}

\objItemVersion{1.0.2}

\objItemHelp{yes}

\objItemTheme{Vector manipulation}

\objItemClass{Arith/Logic/Bitwise, Lists}

\objItemArgs{\ }

  \objListArgBegin
  \objListArgItem{operator}{symbol}{the operator to be applied.
     Recognized operators are sum (`+'), logical AND (`\&\&'), bit-wise AND (`\&'),
     bit-wise OR (`$\mid$'), divide (`/'), exclusive-OR (`$\wedge$'), maximum (`max'),
     minimum (`min'), modulus (`\%'), product (`*'), logical OR (`$\mid\mid$'), or
     subtract (`$-$').
     The result of applying the operator to each element in order (with the previous result being
     used as the left operand and the new element being used as the right operand) are collected
     and returned as a new list.
     For an empty list, the identity element (if defined for the operator), or zero, is returned.}
  \objListArgEnd

\objItemInlet{\ }

  \objListIOBegin
  \objListIOItem{list\textnormal{/}number}{the list to be scanned.
    A single number is treated as a single element list.}
  \objListIOEnd

\objItemOutlet{\ }

  \objListIOBegin
  \objListIOItem{list}{the scan result}
  \objListIOEnd

\objItemCompanion{none}

\objItemStandalone{yes}

\objItemRetainsState{no}

\objItemCompatibility{\MaxName{} 3.x and \MaxName{} 4.x \{OS 9 and OS X\}}

\objItemFat{Fat}

\objItemCommands

\objItemFile

\objItemMessage

\objItemComments

\objEnd{\objNameE{Vscan}}
