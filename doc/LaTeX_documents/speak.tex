\ProvidesFile{speak.tex}[v1.0.3]
\startObject{\objNameS{speak}}{speak}
\index{Themes!Device~interface!speak}
\objPicture{speaksymbol.ps}
\objItemDescription{\objNameD{speak} is an interface to the Macintosh Speech Synthesis Manager,
providing control over pitch, rate, volume and voice.}

\objItemCreated{April 2001}

\objItemVersion{1.0.3}

\objItemHelp{yes}

\objItemTheme{Device interface}

\objItemClass{Devices}

\objItemArgs{none}

\objItemInlet{\ }

  \objListIOBegin
  \objListIOItem{integer\textnormal{/}float\textnormal{/}list\textnormal}{the command input}
  \objListIOEnd

\objItemOutlet{\ }

  \objListIOBegin
  \objListIOItem{bang}{speaking has stopped}

  \objListIOItem{bang}{speaking has paused}

  \objListIOItem{list}{the response to the \objCmdQ{pitch?}, \objCmdQ{rate?},
  \objCmdQ{voice?}, \objCmdQ{voiceMax} or \objCmdQ{volume?} commands.
  Response messages appear as a two element list, starting with one of the symbols `pitch', `rate',
  `voice', `voicemax', `volume' and followed by the appropriate value--an integer value for `voice' and
  `voicemax' and a floating-point value for `pitch', `rate' and `volume'.}

  \objListIOItem{bang}{an error was detected}
  
  \objListIOEnd

\objItemCompanion{none}

\objItemStandalone{yes}

\objItemRetainsState{yes, the Speech Manager settings}

\objItemCompatibility{\MaxName{} 3.x and \MaxName{} 4.x \{OS 9 and OS X\}}

\objItemFat{Fat}

\objItemCommands[]

  \objListCmdBegin
  
  \objListCmdItem{continue}{}
  Cause the Speech Manager to resume speaking if it's paused.
  
  \objListCmdItem{\emph{float}}{}
  Send the given number to the Speech Manager to be spoken immediately.
  
  \objListCmdItem{\emph{integer}}{}
  Send the given number to the Speech Manager to be spoken immediately.
  
  \objListCmdItem{list}{anything}
  Send the given list to the Speech Manager to be spoken immediately.
  
  \objListCmdItem{pause}{}
  Cause the Speech Manager to pause speaking immediately.
  
  \objListCmdItem{pitch}{float}
  Select a new pitch for speaking.
  Higher values correspond to higher frequencies.
  
  \objListCmdItem{pitch?}{}
  Report the speech pitch as a two element list, with the symbol `pitch' as its first element and
  the pitch as a floating-point number as its second element.
  
  \objListCmdItem{rate}{float}
  Select a new rate for speaking.
  The given value is measured (approximately) in terms of words-per-minute.
  Use the \objCmdQ{rate?} command to get the current rate in a form that can be sent back to a
  \objNameX{speak} object.
  
  \objListCmdItem{rate?}{}
  Report the speech rate as a two element list, with the symbol `rate' as its first element and
  the rate as a floating-point number as its second element.
 
  \objListCmdItem{say}{anything}
  Send the given list to the Speech Manager to be spoken immediately.
  This message permits the speaking of an arbitrary list; in particular any list that could be
  confused with a command to the \objNameX{speak} object.
  
  \objListCmdItem{spell}{\textnormal{[}on\textnormal{/}off\textnormal{]}}
  Spelling of the following messages will be enabled (`on'), disabled (`off') or reversed, if no argument is given.
  
  \objListCmdItem{stop}{}
  Cause the Speech Manager to immediately stop speaking.
  
  \objListCmdItem{voice}{integer}
  Select a new voice to be used for speaking.
  Use the \objCmdQ{voice?} command to get the current voice in a form that can be sent back to a
  \objNameX{speak} object.
  
  \objListCmdItem{voice?}{}
  Report the active voice number as a two element list, with the symbol `voice' as its first element and
  the voice number as an integer as its second element.
  
  \objListCmdItem{voicemax?}{}
  Report the maximum voice number as a two element list, with the symbol `voicemax' as its first element and
  the maximum voice number as an integer as its second element.
  
  \objListCmdItem{volume}{float}
  Select a new volume for speaking.
  The given value is between zero (silence) and one (maximum possible volume).
  Use the \objCmdQ{volume?} command to get the current volume in a form that can be sent back to a
  \objNameX{speak} object.
  
  \objListCmdItem{volume?}{}
  Report the speech volume as a two element list, with the symbol `volume' as its first element and
  the volume as a floating-point number as its second element.
 
  \objListCmdEnd

\objItemFile

\objItemMessage

\objItemComments

\objEnd{\objNameE{speak}}
