\ProvidesFile{listType.tex}[v1.0.3]
\startObject{\objNameS{listType}}{listType}
\index{Themes!Miscellaneous!listType}
\objPicture{listTypesymbol.ps}
\objItemDescription{\objNameD{listType} returns a numeric code corresponding to the value it receives.}

\objItemCreated{October 2000}

\objItemVersion{1.0.3}

\objItemHelp{yes}

\objItemTheme{Miscellaneous}

\objItemClass{Lists}

\objItemArgs{none}

\objItemInlet{\ }

  \objListIOBegin
  \objListIOItem{anything}{any value}
  \objListIOEnd

\objItemOutlet{\ }

  \objListIOBegin
  \objListIOItem{integer}{the code for the input (unknown~=~0, non-list~=~1, empty~list~=~2,
      integer~list~=~3, float~list~=~4, numeric~list\emphFootnoteMark~=~5, symbol~list~=~6,
      mixed~list\emphFootnoteMark~=~7, list~with~unknowns\emphFootnoteMark~=~8)}
  \objListIOEnd
     % adjust for the number of footnote marks:
     \addtocounter{footnote}{-2}
     \footnotetext{A numeric list contains both integer and float values.}
     \stepcounter{footnote}
     \footnotetext{A mixed list contains both numeric values and symbols.}
     \stepcounter{footnote}
     \footnotetext{A list with unknowns contains one or more unrecognizable values.}
  
\objItemCompanion{none}

\objItemStandalone{yes}

\objItemRetainsState{no}

\objItemCompatibility{\MaxName{} 3.x and \MaxName{} 4.x \{OS 9 and OS X\}}

\objItemFat{Fat}

\objItemCommands

\objItemFile

\objItemMessage

\objItemComments

\objEnd{\objNameE{listType}}
