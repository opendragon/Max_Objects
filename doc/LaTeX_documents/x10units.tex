\ProvidesFile{x10units.tex}[v1.0.6]
\startObject{\objNameS{x10units}}{x10units}
\index{Themes!Miscellaneous!x10units}
\objPicture{x10unitssymbol.ps}
\objItemDescription{\objNameD{x10units} is an auxiliary object to be used with \objReference{x10} objects.
Its purpose is to permit the use of simple numbers to represent the unit codes used with the
\objName{x10} object, by mapping small integers into the bit patterns needed.}

\objItemCreated{September 1996}

\objItemVersion{1.0.6}

\objItemHelp{no}

\objItemTheme{Miscellaneous}

\objItemClass{Lists}

\objItemArgs{none}

\objItemInlet{\ }

  \objListIOBegin
  \objListIOItem{integer\textnormal{/}list\textnormal{/}bang}{the value to be mapped}
  \objListIOEnd
  
\objItemOutlet{\ }

  \objListIOBegin
  \objListIOItem{integer}{the mapped output of the inlet value, or the previous result
     (if a `bang' is received).
     If the inlet value is a list, the outlet value will be the bit-wise `OR' of the result of
     mapping each list element.}
  \objListIOEnd

\objItemCompanion{works with \objName{x10} objects}

\objItemStandalone{yes}

\objItemRetainsState{no}

\objItemCompatibility{\MaxName{} 3.x and \MaxName{} 4.x \{OS 9 and OS X\}}

\objItemFat{Fat}

\objItemCommands[]

  \objListCmdBegin
  \objListCmdItem{\emph{bang}}{}
  Return the previous result, if any.
  \objListCmdEnd

\objItemFile

\objItemMessage

\objItemComments

\objEnd{\objNameE{x10units}}
