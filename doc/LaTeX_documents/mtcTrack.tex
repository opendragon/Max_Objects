\ProvidesFile{mtcTrack.tex}[v1.0.2]
\startObject{\objNameS{mtcTrack}}{mtcTrack}
\index{Themes!Miscellaneous!mtcTrack}
\objPicture{mtcTracksymbol.ps}
\objItemDescription{\objNameD{mtcTrack} is an auxiliary object to be used with \objReference{mtc} objects.
Its purpose is to convert the raw list of coordinate and pressure data into a series of lines.}

\objItemCreated{April 2001}

\objItemVersion{1.0.2}

\objItemHelp{no}

\objItemTheme{Miscellaneous}

\objItemClass{Lists}

\objItemArgs{\nothing}

  \objListArgBegin
  \objListArgItem{num-lines}{integer}{the maximum number of lines to return}

  \objListArgItem{batch-flag}{(optional) symbol}{whether to output the batch number.
       If the word `batch' appears, the batch number is prepended to each output line.
       By default, the batch number is not output.
       Batch numbers start at zero and are incremented with each set of lines returned, regardless of whether the batch
       number is output.}

  \objListArgItem{index-flag}{(optional) symbol}{whether to output the index of the line, relative to the other lines in the
       batch.
       If the word `index' appears, the index is prepended to each output line, after the batch number, if it's present.
       By default, the index is not output.
       Indices start at zero.}

  \objListArgEnd

\objItemInlet{\nothing}

  \objListIOBegin
  \objListIOItem{list}{the values to be converted, from the companion \objName{mtc} object}
  \objListIOEnd
  
\objItemOutlet{\nothing}

  \objListIOBegin
  \objListIOItem{list}{the recognized lines, in the form of a six, seven or eight element list of numbers,
  consisting of the (optional) batch number, the (optional) line index (which starts at zero), the starting x coordinate,
  the starting y coordinate, the ending x coordinate, the ending y coordinate, the ``velocity'' and the ``force''.
  The last two elements are estimates of the line's characteristics.}

  \objListIOItem{bang}{the last recognized line has been sent}
  
  \objListIOItem{integer}{the number of lines recognized.
  This value is output before the lines are output, so that the lines can be collected and processed in a batch.}  
  
  \objListIOItem{bang}{an error was detected}
  
  \objListIOEnd

\objItemCompanion{works with \objName{mtc} objects}

\objItemStandalone{yes}

\objItemRetainsState{yes, the previously identified points}

\objItemCompatibility{\MaxName{} 3.x and \MaxName{} 4.x \{OS 9 and OS X\}}

\objItemFat{Fat}

\objItemCommands[]

  \objListCmdBegin
  
  \objListCmdItem{batch}{\textnormal{[}on\textnormal{/}off\textnormal{]}}
  Output of the batch number with each line will be enabled (`on'), disabled (`off') or reversed, if no argument is given.
  
  \objListCmdItem{clear}{}
  Reset the number of previously identified points, so that the next set of points received will be considered
  the start of a set of new lines.

  \objListCmdItem{index}{\textnormal{[}on\textnormal{/}off\textnormal{]}}
  Output of the line number index with each line will be enabled (`on'), disabled (`off') or reversed, if no argument is given.

  \objListCmdItem{threshold}{distance}
  Set the maximum distance that can separate the starting and ending points of a recognized line.
  If \objCmdArg{distance} is negative, there is no maximum.
  Use of this command reduces the thrashing that occurs when the matching points of lines are not sufficiently separated to
  permit unambiguous determination of the recognized lines.
  
  \objListCmdEnd

\objItemFile

\objItemMessage

\objItemComments[The \objNameX{mtcTrack} object was designed to assist in working with the \objName{mtc} object.
The maximum number of lines to return should be close in value to the the maximum number of
hot spots specified for the companion \objName{mtc} object.]

\objEnd{\objNameE{mtcTrack}}

% $Log: mtcTrack.tex,v $
% Revision 1.6  2006/07/20 04:47:53  churchoflambda
% Re-added the files to record their changes.
%
% Revision 1.4  2005/08/02 15:07:09  churchoflambda
% Added CVS tags; add rail diagrams for pfsm, map1d, map2d, map3d and listen.
%

