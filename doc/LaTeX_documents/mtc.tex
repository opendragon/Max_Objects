\ProvidesFile{mtc.tex}[v1.0.7]
\startObject{\objNameS{mtc}}{mtc}
\index{Themes!Device~interface!mtc}
\objPicture{mtcsymbol.ps}
\objItemDescription{\objNameD{mtc} is an interface to the \companyReference{http://www.tactex.com}{Tactex Controls} multi-touch controller (MTC).
It sends commands to a \objReference{serialX} object, which controls the serial port that the MTC is attached to, and responds to data returned
from the MTC via the \objName{serialX} object.}

\objItemCreated{December 1999}

\objItemVersion{1.0.7}

\objItemHelp{no}

\objItemTheme{Device interface}

\objItemClass{Devices}

\objItemArgs{\nothing}

  \objListArgBegin
  \objListArgItem{num-spots}{integer}{the maximum number of hot spots to return}

  \objListArgItem{map-file}{symbol}{the map file that contains the coordinates of the sensor points for the MTC device}

  \objListArgItem{norm-file}{symbol}{the normalization file for the pressure values for the MTC device}

  \objListArgItem{mode}{(optional) symbol}{the initial processing mode (raw or cooked) that is to be used.
       By default, the mode is cooked.}

  \objListArgItem{order}{(optional) symbol}{the initial sort order (pressure, x, or y) that is to be used to prioritize the hot spots.
       By default, the hot spots are unordered.}

  \objListArgItem{poll-rate}{(optional) integer}{the rate (in milliseconds) at which the companion \objName{serialX} object is polled
       via a sample request.
       The default rate is 100 milliseconds between sample requests.}

  \objListArgEnd

\objItemInlet{\nothing}

  \objListIOBegin
  \objListIOItem{list}{the command channel}
  
  \objListIOItem{anything}{the output of the companion \objName{serialX} object}
  
  \objListIOEnd

\objItemOutlet{\nothing}

  \objListIOBegin
  \objListIOItem{list}{the detected hot spots in the form of floating point triples (x coordinate, y coordinate,
  pressure)}
  
  \objListIOItem{integer}{the data has started (`0') or ended (`1').
  The `0' to `1' transition preceeds the data from the MTC and the `1' to `0' transition follows the data from the MTC.}
  
  \objListIOItem{bang}{the command has completed}

  \objListIOItem{bang}{the sample request to send to the companion \objName{serialX} object}

  \objListIOItem{anything}{the data to send to the companion \objName{serialX} object}

  \objListIOItem{bang}{the \objCmdQ{chunk} request to send to the companion \objName{serialX} object, via a message object}
  
  \objListIOItem{bang}{an error was detected}
  
  \objListIOEnd

\objItemCompanion{works with \objName{serialX} objects (not \objNameS{serial}) and can be connected to
\objReference{mtcTrack} objects}

\objItemStandalone{yes}

\objItemRetainsState{yes}

\objItemCompatibility{\MaxName{} 3.x and \MaxName{} 4.x \{OS 9 and OS X\}}

\objItemFat{Fat}

\objItemCommands[]

  \objListCmdBegin
  
  \objListCmdItem{describe}{}
  Read the MTC descriptor string from the device (if it hasn't already been read) and send the string to the \MaxName{} window.

  \objListCmdItem{mode}{symbol}
  Change the processing mode (where \objCmdArg{symbol} is `raw' or `cooked') that is to be used.

  \objListCmdItem{order}{symbol}
  Change the sort order (where \objCmdArg{symbol} is `pressure', `x', or `y') that is to be used to prioritize the hot spots.

  \objListCmdItem{ping}{}
  Perform a `ping' of the MTC device to verify connectivity.

  \objListCmdItem{start}{\textnormal{[}normal\textnormal{/}compressed\textnormal{]}}
  Start sampling the MTC device for data (hot spots or raw pressures).
  If `compressed' is requested, the compressed form of data communication is used with the MTC device;
  if `normal' is specified, or no argument is given, then the uncompressed form of data communication is used with the MTC device.

  \objListCmdItem{stop}{}
  Stop sampling the MTC device for data (hot spots or raw pressures).
  
  \objListCmdItem{threshold}{new-level}
  Set the pressure threshold for hot spots to \objCmdArg{new-level}.
  Sensors with values above the pressure threshold are candidates for the hot spot list.

  \objListCmdItem{train}{\textnormal{[}start\textnormal{/}stop\textnormal{]}}
  Start or stop the `training' mode of the \objNameX{mtc} object, where the high and low pressure levels for each sensor are
  determined to establish the dynamic range of the sensor.

  \objListCmdItem{verbose}{\textnormal{[}on\textnormal{/}off\textnormal{]}}
  Tracing to the \MaxName{} window of the messages sent will be enabled (`on'), disabled (`off') or reversed, if no argument is given.
  
  \objListCmdEnd

\objItemFile

\objItemMessage

\objItemComments[Figure~\objImageReference{diagram:mtcconnect} shows how to connect an \objNameX{mtc} object to a
\objName{serialX} object.
Figure~\objImageReference{diagram:mtcstate} is a state diagram for \objNameX{mtc} objects.]
\objDiagram{mtcconnections.ps}{mtcconnect}{Connecting an \objNameX{mtc} object to a \objName{serialX} object}
\objDiagram{mtcstate.ps}{mtcstate}{State diagram for \objNameX{mtc} objects}

\objEnd{\objNameE{mtc}}

% $Log: mtc.tex,v $
% Revision 1.3  2005/08/02 15:07:09  churchoflambda
% Added CVS tags; add rail diagrams for pfsm, map1d, map2d, map3d and listen.
%

