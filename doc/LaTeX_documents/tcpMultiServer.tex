\ProvidesFile{tcpMultiServer.tex}[v1.0.8]
\startObject{\objNameS{tcpMultiServer}}{tcpMultiServer}
\index{Themes!TCP/IP!tcpMultiServer}
\objPicture{tcpMultiServersymbol.ps}
\objItemDescription{\objNameD{tcpMultiServer} is an interface to the TCP/IP stack on a Macintosh,
providing an endpoint server to communicate with one or more \objReference{tcpClient} objects.}

\objItemCreated{October 2000}

\objItemVersion{1.0.8}

\objItemHelp{yes}

\objItemTheme{TCP/IP}

\objItemClass{Devices}

\objItemArgs{\ }

  \objListArgBegin
  \objListArgItem{port}{(optional) integer}{the number of the port to communicate on.
      The default port is 65535, which is also the maximum acceptable port number}

  \objListArgItem{clients}{(optional) integer}{the maximum number of clients that will be supported.
      The maximum that can be specified is 100 and the default is 5.}

  \objListArgItem{buffers}{(optional) integer}{the total number of receive buffers to use.
      The default number of buffers is 25 times the maximum number of clients}
      
  \objListArgEnd

\objItemInlet{\ }

  \objListIOBegin
  \objListIOItem{list}{the command input}
  \objListIOEnd

\objItemOutlet{\ }

  \objListIOBegin
  \objListIOItem{list}{the status or response.
       Status messages (triggered by a \objCmdQ{bang} or \objCmdQ{status} command) appear as a
       five-element list, starting with the symbol `status'; response messages appear as a list,
       starting with the symbol `reply' and `self' messages appear as a two-element list, starting
       with the symbol `self'.
       The second element of the status or response message is the identifier of the client
       connection involved.}
  \objListIOEnd

\objItemCompanion{none}

\objItemStandalone{no, works with multiple \objName{tcpClient} objects on the same computer or other
computers that are reachable via a TCP/IP network}

\objItemRetainsState{yes}

\objItemCompatibility{\MaxName{} 3.x and \MaxName{} 4.x \{OS 9 and OS X\}}

\objItemFat{PPC-only}

\objItemCommands[]

  \objListCmdBegin

  \objListCmdItem{\emph{bang}}{}
  Report the general state of the communication as a five-element list, with the symbol `status' as
  its first element.
  The second element of the list is the value `0', the third element is the general state of the
  communication (`unbound', `bound', `listening' or `unknown'), the fourth element is the number of
  active client connections and the fifth element is the port number that is being listened to.

  \objListCmdItem{disconnect}{client}
  Close any existing connection to the \objName{tcpClient} object with the given identifier,
  if \objCmdArg{client} is non-zero, or close all existing connections if \objCmdArg{client} is zero.

  \objListCmdItem{listen}{on\textnormal{/}off}
  Start listening on the communication port (`on') or stop listening (`off').
  The command is only accepted if the \objNameX{tcpMultiServer} object is in an appropriate state---
  `listening' if `off' and `bound' if `on'.

  \objListCmdItem{port}{integer}
  Set the number of the port to communicate on.

  \objListCmdItem{self}{}
  Returns the IP address of this object as a two-element list, with the symbol `self' as its first
  element.
  
  \objListCmdItem{send}{client anything}
  Send the arguments to the connected \objName{tcpClient} objects with the given identifier,
  if \objCmdArg{client} is non-zero, or send the arguments to all existing connections if
  \objCmdArg{client} is zero (a broadcast).
  This message permits the transmission of an arbitrary list.

  \objListCmdItem{status}{\textnormal{[}client\textnormal{]}}
  Report either the state of a specific client communication, if \objCmdArg{client} is non-zero
  (a valid identifier), or the general state of the communication, if \objCmdArg{client} is zero,
  as a five-element list, with the symbol `status' as its first element.
  The second element of the list is \objCmdArg{client}, the third element is the general state of
  the communication (`unbound', `bound', `listening' or `unknown') or the specific client
  communication (`connecting', `connected', `disconnecting' or `unknown').
  If \objCmdArg{client} is zero, the fourth element is the number of active client connections and
  the fifth element is the port number that is being listened to.
  If \objCmdArg{client} is non-zero and the state is `connected', the fourth element is the IP
  address of the client, expressed as a symbol in `dotted' notation and the fifth element is the
  port of the client.
  The default for \objCmdArg{client} is 0, which requests the general state of the communication.

  \objListCmdItem{verbose}{\textnormal{[}on\textnormal{/}off\textnormal{]}}
  Communication tracing to the \MaxName{} window will be enabled (`on'), disabled (`off') or
  reversed, if no argument is given.
  
  \objListCmdEnd

\objItemFile

\objItemMessage[see \objName{tcpClient}]

\objItemComments[Once a communication path is established between a \objName{tcpClient} object and
a \objNameX{tcpMultiServer} object, either object can send messages---the path is full-duplex.
The client identifier used with the commands \objCmdQ{disconnect}, \objCmdQ{send} and \objCmdQ{status}
and returned as part of the `reply' and `status' messages is a positive integer between 1 and the
maximum number of clients specified when the \objNameX{tcpMultiServer} object was created.
Note that the identifier exists until the connection is closed with a \objCmdQ{disconnect} command
(sent to the \objNameX{tcpMultiServer} object or the connected \objName{tcpClient} object) and the
identifier may be reassigned to a subsequent connection.
Note as well that only there can only be one \objNameX{tcpMultiServer} or \objName{tcpServer} object
for any given port.
Figure~\objImageReference{diagram:tcpmultiserverstate} is a state diagram for
\objNameX{tcpMultiServer} objects.]
\objDiagram{tcpMultiServerstate.ps}{tcpmultiserverstate}{State diagram for \objNameX{tcpMultiServer} objects}

\objEnd{\objNameE{tcpMultiServer}}
