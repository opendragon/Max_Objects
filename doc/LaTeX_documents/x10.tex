\ProvidesFile{x10.tex}[v1.0.8]
\startObject{\objNameS{x10}}{x10}
\index{Themes!Device~interface!x10}
\objPicture{x10symbol.ps}
\objItemDescription{\objNameD{x10} is an interface to the X-10 controller originally
from X-10 Incorporated.
It sends commands to a \objReference{serialX} object, which controls the serial
port that the X-10 is attached to, and responds to data returned from the X-10 via the
\objName{serialX} object.}

\objItemCreated{September 1996}

\objItemVersion{1.0.8}

\objItemHelp{no}

\objItemTheme{Device interface}

\objItemClass{Devices}

\objItemArgs{\nothing}

  \objListArgBegin
  \objListArgItem{kind}{(optional) symbol}{the type of X-10 controller to communicate with.
      The acceptable values are `cm11' and `cp290'.
      The default kind is `cm11'.}
      
  \objListArgItem{poll-rate}{(optional) integer}{the rate (in milliseconds) at which the companion
      \objName{serialX} object is polled via a sample request.
      The default rate is 100 milliseconds between sample requests.}
  \objListArgEnd

\objItemInlet{\nothing}

  \objListIOBegin
  \objListIOItem{list}{the command channel}

  \objListIOItem{integer}{the feedback from the \objName{serialX} object}
  
  \objListIOEnd

\objItemOutlet{\nothing}

  \objListIOBegin
  \objListIOItem{integer}{the commands to the \objName{serialX} object}

  \objListIOItem{bang}{the sample request to the \objName{serialX} object}

  \objListIOItem{integer}{the base house-code}

  \objListIOItem{bang}{the command has completed}

  \objListIOItem{integer}{the device status (sent as a result of each command)}

  \objListIOItem{integer}{the day number}

  \objListIOItem{integer}{the hour number}

  \objListIOItem{integer}{the minute number}

  \objListIOItem{bang}{an error was detected}
  
  \objListIOEnd

\objItemCompanion{works with \objName{serialX} objects, can be attached to \objReference{x10units} objects}

\objItemStandalone{yes}

\objItemRetainsState{yes}

\objItemCompatibility{\MaxName{} 3.x and \MaxName{} 4.x \{OS 9 and OS X\}}

\objItemFat{Fat}

\objItemCommands[]

  \objListCmdBegin

  \objListCmdItem{clear!}{eventNumber}
  Remove the given event, \objCmdArg{eventNumber}, where \objCmdArg{eventNumber} is an integer
  between 0 and 127.

  \objListCmdItem{clock!}{}
  Set the clock of the X-10 device to match the Macintosh time-of-day clock.

  \objListCmdItem{dim}{house-code map \textnormal{[}level\textnormal{]}}
  Given a set of unit codes for the devices controlled by the X-10 device, \objCmdArg{map}, and the
  name of the X-10 device, \objCmdArg{house-code}, immediately dim the specified devices to the value
  of \objCmdArg{level}, which is an integer between 0 and 15.
  The set of unit codes is obtained by passing the list of device numbers through the
  \objName{x10units} object.
  The name of the X-10 device is a single character symbol between `A' and `P'.
  If \objCmdArg{level} is not given, it is assumed to be zero.

  \objListCmdItem{events?}{}
  Interrogates the X-10 device for its event data, but does nothing with the results.

  \objListCmdItem{everyday!}{house-code map eventNumber hourMinute function
     \textnormal{[}level\textnormal{]}}
  Given a set of unit codes for the devices controlled by the X-10 device, \objCmdArg{map}, and
  the name of the X-10 device, \objCmdArg{house-code}, an event number, \objCmdArg{eventNumber},
  the encoded time, \objCmdArg{hourMinute}, the operation to perform, \objCmdArg{function}
  (either `on', `off' or `dim') and the dimming level, \objCmdArg{level}, record the event in the
  X-10 device as a `normal' event for activation at the given time, every day of the week.
  The set of unit codes is obtained by passing the list of device numbers through the
  \objName{x10units} object.
  The name of the X-10 device is a single character symbol between `A' and `P'.
  The event number is an integer between 0 and 127.
  The time is encoded by multiplying the desired hour by 60 and adding the minutes; it is effectively
  the number of minutes after midnight.
  The dimming level is an integer between 0 and 15.

  \objListCmdItem{graphics!}{object-number \textnormal{[}object-data\textnormal{]}}
  Set the graphics data of the X-10 device to the two given integer values.
  There is no current use for this information.
  If \objCmdArg{object-data} is not given, it is assumed to be zero.

  \objListCmdItem{graphics?}{}
  Interrogates the X-10 device for its icon data, but does nothing with the results.

  \objListCmdItem{housecode!}{letter}
  Set the name of the X-10 device to the given value, \objCmdArg{letter}, where the value is a
  single character symbol between `A' and `P'.

  \objListCmdItem{housecode?}{}
  Return the current time (day, hour and minute) and the X-10 house-code via the corresponding outlets.

  \objListCmdItem{off}{house-code map}
  Given a set of unit codes for the devices controlled by the X-10 device, \objCmdArg{map}, and the
  name of the X-10 device, \objCmdArg{house-code}, immediately turn off the specified devices.
  The set of unit codes is obtained by passing the list of device numbers through the
  \objName{x10units} object.
  The name of the X-10 device is a single character symbol between `A' and `P'.

  \objListCmdItem{on}{house-code map}
  Given a set of unit codes for the devices controlled by the X-10 device, \objCmdArg{map}, and
  the name of the X-10 device, \objCmdArg{house-code}, immediately turn on the specified devices.
  The set of unit codes is obtained by passing the list of device numbers through the
  \objName{x10units} object.
  The name of the X-10 device is a single character symbol between `A' and `P'.

  \objListCmdItem{kind}{device-type}
  Change the protocol to match the given \objCmdArg{device-type}.
  The \objCmdArg{device-type} is either `cm11' or `cp290'.
  
  \objListCmdItem{normal!}{house-code map eventNumber dayMap hourMinute function
     \textnormal{[}level\textnormal{]}}
  Given a set of unit codes for the devices controlled by the X-10 device, \objCmdArg{map}, and
  the name of the X-10 device, \objCmdArg{house-code}, an event number, \objCmdArg{eventNumber},
  a set of days, \objCmdArg{dayMap}, the encoded time, \objCmdArg{hourMinute}, the operation to
  perform, \objCmdArg{function} (either `on', `off' or `dim') and the dimming level,
  \objCmdArg{level}, record the event in the X-10 device as a `normal' event for activation at
  the given time on the given days.
  The set of unit codes is obtained by passing the list of device numbers through the
  \objName{x10units} object.
  The name of the X-10 device is a single character symbol between `A' and `P'.
  The event number is an integer between 0 and 127.
  The set of days is an integer between 0 and 127, representing which days the event is to occur on,
  with a single bit corresponding to each day of the week.
  The time is encoded by multiplying the desired hour by 60 and adding the minutes; it is
  effectively the number of minutes after midnight.
  The dimming level is an integer between 0 and 15.

  \objListCmdItem{reset}{}
  Send a diagnostic command to the X-10 device, forcing a reset of the device.

  \objListCmdItem{security!}{house-code map eventNumber dayMap hourMinute function
     \textnormal{[}level\textnormal{]}}
  Given a set of unit codes for the devices controlled by the X-10 device, \objCmdArg{map}, and the
  name of the X-10 device, \objCmdArg{house-code}, an event number, \objCmdArg{eventNumber},
  a set of days, \objCmdArg{dayMap}, the encoded time, \objCmdArg{hourMinute}, the operation to
  perform, \objCmdArg{function} (either `on', `off' or `dim') and the dimming level,
  \objCmdArg{level}, record the event in the X-10 device as a `secure' event for activation at the
  given time on the given days.
  The set of unit codes is obtained by passing the list of device numbers through the
  \objName{x10units} object.
  The name of the X-10 device is a single character symbol between `A' and `P'.
  The event number is an integer between 0 and 127.
  The set of days is an integer between 0 and 127, representing which days the event is to occur on,
  with a single bit corresponding to each day of the week.
  The time is encoded by multiplying the desired hour by 60 and adding the minutes; it is
  effectively the number of minutes after midnight.
  The dimming level is an integer between 0 and 15.

  \objListCmdItem{today!}{house-code map eventNumber hourMinute function
      \textnormal{[}level\textnormal{]}}
  Given a set of unit codes for the devices controlled by the X-10 device, \objCmdArg{map}, and
  the name of the X-10 device, \objCmdArg{house-code}, an event number, \objCmdArg{eventNumber},
  the encoded time, \objCmdArg{hourMinute}, the operation to perform, \objCmdArg{function}
  (either `on', `off' or `dim') and the dimming level, \objCmdArg{level}, record the event in the
  X-10 device as a `normal' event for activation at the given time today.
  The set of unit codes is obtained by passing the list of device numbers through the
  \objName{x10units} object.
  The name of the X-10 device is a single character symbol between `A' and `P'.
  The event number is an integer between 0 and 127.
  The time is encoded by multiplying the desired hour by 60 and adding the minutes; it is
  effectively the number of minutes after midnight.
  The dimming level is an integer between 0 and 15.

  \objListCmdItem{tomorrow!}{house-code map eventNumber hourMinute function
     \textnormal{[}level\textnormal{]}}
  Given a set of unit codes for the devices controlled by the X-10 device, \objCmdArg{map}, and
  the name of the X-10 device, \objCmdArg{house-code}, an event number, \objCmdArg{eventNumber},
  the encoded time, \objCmdArg{hourMinute}, the operation to perform, \objCmdArg{function}
  (either `on', `off' or `dim') and the dimming level, \objCmdArg{level}, record the event in
  the X-10 device as a `normal' event for activation at the given time tomorrow.
  The set of unit codes is obtained by passing the list of device numbers through the
  \objName{x10units} object.
  The name of the X-10 device is a single character symbol between `A' and `P'.
  The event number is an integer between 0 and 127.
  The time is encoded by multiplying the desired hour by 60 and adding the minutes; it is
  effectively the number of minutes after midnight.
  The dimming level is an integer between 0 and 15.

  \objListCmdItem{xreset}{}
  Clear the internal state of the \objNameX{x10} object, without waiting for completion of pending
  commands.
  
  \objListCmdEnd

\objItemFile

\objItemMessage

\objItemComments[Figure~\objImageReference{diagram:x10connect} shows how to connect an
\objNameX{x10} object to a \objName{serialX} object.]
\objDiagram{x10connections.ps}{x10connect}{Connecting an \objNameX{x10} object to a \objName{serialX} object}

\objEnd{\objNameE{x10}}
