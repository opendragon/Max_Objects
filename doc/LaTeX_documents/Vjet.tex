\ProvidesFile{Vjet.tex}[v1.0.2]
\startObject{\objNameS{Vjet}}{Vjet}
\index{Themes!Vector~manipulation!Vjet}
\index{Vectors!Dyadic~operations!Vjet}
\objPicture{Vjetsymbol.ps}
\objItemDescription{\objNameD{Vjet} takes as input a list and divides it into a series of
fixed-size, shorter, lists.
It is similar to the \compLang{APL} `reshape' operator ($\rho$)\index{APL!reshape}.}

\objItemCreated{July 2000}

\objItemVersion{1.0.2}

\objItemHelp{yes}

\objItemTheme{Vector manipulation}

\objItemClass{Lists}

\objItemArgs{\nothing}

  \objListArgBegin
  \objListArgItem{how-many}{integer}{the size of the output fragments}
  \objListArgEnd

\objItemInlet{\nothing}

  \objListIOBegin
  \objListIOItem{bang\textnormal{/}list}{the input to be processed}

  \objListIOItem{integer}{the size of the output fragments.
      This replaces the initial argument.}

  \objListIOEnd

\objItemOutlet{\nothing}

  \objListIOBegin
  \objListIOItem{list}{the input after segmentation, or the previous result (if a `bang' is received)}
  \objListIOEnd

\objItemCompanion{none}

\objItemStandalone{yes}

\objItemRetainsState{yes, the size of the output fragments}

\objItemCompatibility{\MaxName{} 3.x and \MaxName{} 4.x \{OS 9 and OS X\}}

\objItemFat{Fat}

\objItemCommands[]

  \objListCmdBegin
  \objListCmdItem{\emphcorr{bang}}{}
  Return the previous result, if any.
  \objListCmdEnd

\objItemFile

\objItemMessage

\objItemComments[The \objNameX{Vjet} object was designed to assist in working with the \objReference{mtc} object,
which returns triples of values in the form of a long list.
If the input list is not exactly divisible by the size of the output fragments, the last fragment
will be shorter than all the earlier fragments.]

\objEnd{\objNameE{Vjet}}

% $Log: Vjet.tex,v $
% Revision 1.5  2006/07/20 04:47:49  churchoflambda
% Re-added the files to record their changes.
%
% Revision 1.3  2005/08/02 15:07:03  churchoflambda
% Added CVS tags; add rail diagrams for pfsm, map1d, map2d, map3d and listen.
%
