\ProvidesFile{gcd.tex}[v1.0.3]
\startObject{\objNameS{gcd}}{gcd}
\index{Themes!Miscellaneous!gcd}
\objPicture{gcdsymbol.ps}
\objItemDescription{\objNameD{gcd} calculates the greatest common divisor of two numbers.}

\objItemCreated{October 1998}

\objItemVersion{1.0.3}

\objItemHelp{yes}

\objItemTheme{Miscellaneous}

\objItemClass{Arith/Logic/Bitwise}

\objItemArgs{none}

\objItemInlet{\nothing}

  \objListIOBegin
  \objListIOItem{bang\textnormal{/}integer}{the first number to use}

  \objListIOItem{integer}{the second number to use}
  \objListIOEnd

\objItemOutlet{\nothing}

  \objListIOBegin
  \objListIOItem{integer}{the greatest common divisor of the two numbers,
      or the previous result (if a \objCmdQ{bang} is received)}
  \objListIOEnd

\objItemCompanion{none}

\objItemStandalone{yes}

\objItemRetainsState{yes, the previous number input}

\objItemCompatibility{\MaxName{} 3.x and \MaxName{} 4.x \{OS 9 and OS X\}}

\objItemFat{Fat}

\objItemCommands[]

  \objListCmdBegin
  \objListCmdItem{\emphcorr{bang}}{}
  Return the previous result, if any.
  \objListCmdEnd

\objItemFile

\objItemMessage

\objItemComments[In mathematical terms: $y \gets \gcd(x_1,x_2)$, where $x_1$ and $x_2$ are the
inlet values and $y$ is the outlet result.]

\objEnd{\objNameE{gcd}}

% $Log: gcd.tex,v $
% Revision 1.5  2006/07/20 04:47:51  churchoflambda
% Re-added the files to record their changes.
%
% Revision 1.3  2005/08/02 15:07:09  churchoflambda
% Added CVS tags; add rail diagrams for pfsm, map1d, map2d, map3d and listen.
%

