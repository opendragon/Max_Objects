\ProvidesFile{sysLogger.tex}[v1.0.3]
\startObject{\objNameS{sysLogger}}{sysLogger}
\index{Themes!Miscellaneous!sysLogger}
\objPicture{sysLoggersymbol.ps}
\objItemDescription{\objNameD{sysLogger} writes it's input to the syslogd facility for Mac OS 9, available from
\companyReference{http://www.classicalguitar.net/brian/apps/syslogd}{Brian Bergstrand} or to the native Mac OS X facility.}

\objItemCreated{March 2002}

\objItemVersion{1.0.3}

\objItemHelp{yes}

\objItemTheme{Miscellaneous}

\objItemClass{Miscellaneous}

\objItemArgs{\nothing}

  \objListArgBegin
  \objListArgItem{level}{(optional) symbol}{the logging level to be used.
  		The acceptable values are:  `emergency', `alert', `critical', `error', `warning', `notice', `info' and `debug'.
        The default level is `info'.}
  \objListArgEnd

\objItemInlet{\nothing}

  \objListIOBegin
  \objListIOItem{anything}{the input to be processed}
  \objListIOEnd

\objItemOutlet{none}

\objItemCompanion{none}

\objItemStandalone{yes}

\objItemRetainsState{no}

\objItemCompatibility{\MaxName{} 3.x and \MaxName{} 4.x}

\objItemFat{Fat}

\objItemCommands

\objItemFile

\objItemMessage

\objItemComments[Use of the \objNameX{sysLogger} object in Mac OS X environment may require additions to the file /etc/syslog.conf;
the output can be viewed using the standard Console application.]

\objEnd{\objNameE{sysLogger}}

% $Log: sysLogger.tex,v $
% Revision 1.5  2006/07/20 04:47:54  churchoflambda
% Re-added the files to record their changes.
%
% Revision 1.3  2005/08/02 15:07:09  churchoflambda
% Added CVS tags; add rail diagrams for pfsm, map1d, map2d, map3d and listen.
%
