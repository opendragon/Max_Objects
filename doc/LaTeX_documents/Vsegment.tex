\ProvidesFile{Vsegment.tex}[v1.0.4]
\startObject{\objNameS{Vsegment}}{Vsegment}
\index{Themes!Vector~manipulation!Vsegment}
\index{Vectors!Dyadic~operations!Vsegment}
\objPicture{Vsegmentsymbol.ps}
\objItemDescription{\objNameD{Vsegment} is used to extract a portion of a list.}

\objItemCreated{July 2000}

\objItemVersion{1.0.4}

\objItemHelp{yes}

\objItemTheme{Vector manipulation}

\objItemClass{Lists}

\objItemArgs{\nothing}

  \objListArgBegin
  \objListArgItem{start}{integer}{the starting element to select from the list.
     A positive number indicates that the selection starts from the beginning of the list, while a
     negative number indicates that the selection starts from the end of the list, with the last
     element having the position of `$-1$'.}

  \objListArgItem{how-many}{integer}{the number of elements to select from the list.
     A positive number indicates that the selection extends from the starting element towards the
     end of the list, while a negative number indicates that the selection extends from the
     starting element towards the beginning of the list.}
  \objListArgEnd

\objItemInlet{\nothing}

  \objListIOBegin
  \objListIOItem{bang\textnormal{/}list}{the list to be reduced}

  \objListIOItem{integer}{the starting element to select from the list.
     This replaces the initial argument \objIOType{start}.}
  
  \objListIOItem{integer}{the number of elements to select from the list.
     This replaces the initial argument \objIOType{how-many}.}
  \objListIOEnd

\objItemOutlet{\nothing}

  \objListIOBegin
  \objListIOItem{list}{the reduced list, or the previous result (if a `bang' is received)}
  \objListIOEnd

\objItemCompanion{none}

\objItemStandalone{yes}

\objItemRetainsState{yes, the starting element and the number of elements}

\objItemCompatibility{\MaxName{} 3.x and \MaxName{} 4.x \{OS 9 and OS X\}}

\objItemFat{Fat}

\objItemCommands[]

  \objListCmdBegin
  \objListCmdItem{\emphcorr{bang}}{}
  Return the previous result, if any.
  \objListCmdEnd

\objItemFile

\objItemMessage

\objItemComments

\objEnd{\objNameE{Vsegment}}
