\ProvidesFile{Vsin.tex}[v1.0.2]
\startObject{\objNameS{Vsin}}{Vsin}
\index{Themes!Miscellaneous!Vsin}
\index{Themes!Vector~manipulation!Vsin}
\index{Vectors!Monadic~operations!Vsin}
\objPicture{Vsinsymbol.ps}
\objItemDescription{\objNameD{Vsin} calculates the sine of the input
(either a list or a single number).}

\objItemCreated{May 2001}

\objItemVersion{1.0.2}

\objItemHelp{yes}

\objItemTheme{Miscellaneous}

\objItemClass{Arith/Logic/Bitwise, Lists}

\objItemArgs{none}

\objItemInlet{\nothing}

  \objListIOBegin
  \objListIOItem{anything}{the input to be processed}
  \objListIOEnd

\objItemOutlet{\nothing}

  \objListIOBegin
  \objListIOItem{anything}{the input after calculating the sine, or the previous result
      (if a `bang' is received)}
  \objListIOEnd

\objItemCompanion{none}

\objItemStandalone{yes}

\objItemRetainsState{no}

\objItemCompatibility{\MaxName{} 3.x and \MaxName{} 4.x \{OS 9 and OS X\}}

\objItemFat{Fat}

\objItemCommands[]

  \objListCmdBegin
  \objListCmdItem{\emphcorr{bang}}{}
  Return the previous result, if any.
  \objListCmdEnd

\objItemFile

\objItemMessage

\objItemComments[In mathematical terms: $y_i \gets \sin{x_i}$, where $x = x_1,x_2,\dots,x_n$ is the inlet value and
$y = y_1,y_2,\dots,y_n$ is the outlet result.]

\objEnd{\objNameE{Vsin}}

% $Log: Vsin.tex,v $
% Revision 1.3  2005/08/02 15:07:04  churchoflambda
% Added CVS tags; add rail diagrams for pfsm, map1d, map2d, map3d and listen.
%
