\ProvidesFile{Vtokenize.tex}[v1.0.0]
\startObject{\objNameS{Vtokenize}}{Vtokenize}
\index{Themes!Vector~manipulation!Vtokenize}
\objItemDescription{\objNameD{Vtokenize} is used to partition a list of numbers into a sequence of sublists,
separated by `noise' numbers in the original list.}

\objItemCreated{June 2003}

\objItemVersion{1.0.0}

\objItemHelp{yes}

\objItemTheme{Vector manipulation}

\objItemClass{Arith/Logic/Bitwise, Lists}

\objItemArgs{\nothing}

  \objListArgBegin
  \objListArgItem{separator1}{integer}{a number that indicates the end of a sublist.
    Only non-zero numbers will be recognized.}
  \objListArgItem{separator2}{(optional) integer}{another number that indicates the end of a sublist.}
  \objListArgItem{separator3}{(optional) integer}{another number that indicates the end of a sublist.}
  \objListArgItem{separator4}{(optional) integer}{another number that indicates the end of a sublist.}
  \objListArgItem{separator5}{(optional) integer}{another number that indicates the end of a sublist.} 
  \objListArgEnd

\objItemInlet{\nothing}

  \objListIOBegin
  \objListIOItem{integer\textnormal{/}list\textnormal{/}bang}{the list to be processed.
     A single number is treated as a single element list.}
  \objListIOEnd

\objItemOutlet{\nothing}

  \objListIOBegin
  \objListIOItem{list}{the generated sublists}

  \objListIOItem{bang}{the last sublist was generated}
  
  \objListIOEnd

\objItemCompanion{none}

\objItemStandalone{yes}

\objItemRetainsState{yes, the separator numbers}

\objItemCompatibility{\MaxName{} 4.x \{OS 9 and OS X\}}

\objItemFat{PPC-only}

\objItemCommands[]

  \objListCmdBegin
  \objListCmdItem{\emphcorr{bang}}{}
  Return the previous sequence of sublists, if any.
  \objListCmdEnd

\objItemFile

\objItemMessage

\objItemComments

\objEnd{\objNameE{Vtokenize}}
