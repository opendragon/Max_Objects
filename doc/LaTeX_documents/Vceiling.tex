\ProvidesFile{Vceiling.tex}[v1.0.5]
\startObject{\objNameS{Vceiling}}{Vceiling}
\index{Themes!Miscellaneous!Vceiling}
\index{Themes!Vector~manipulation!Vceiling}
\index{Vectors!Monadic~operations!Vceiling}
\objPicture{Vceilingsymbol.ps}
\objItemDescription{\objNameD{Vceiling} calculates the smallest integer greater than the value
given (either a list or a single number).}

\objItemCreated{November 2000}

\objItemVersion{1.0.5}

\objItemHelp{yes}

\objItemTheme{Miscellaneous}

\objItemClass{Arith/Logic/Bitwise, Lists}

\objItemArgs{\ }

  \objListArgBegin
  \objListArgItem{mode}{(optional) symbol}{either `f', `i' or `m' to indicate whether the output
        is to be floating-point values only, integer values only, or mixed values.
        Mixed values are floating-point if the input is floating-point and integer if the input is
        integer.
        The default is `m'.}
  
  \objListArgEnd

\objItemInlet{\ }

  \objListIOBegin
  \objListIOItem{anything}{the input to be processed}
  \objListIOEnd

\objItemOutlet{\ }

  \objListIOBegin
  \objListIOItem{anything}{the input after calculating the ceiling, or the previous result
      (if a `bang' is received)}
  \objListIOEnd

\objItemCompanion{none}

\objItemStandalone{yes}

\objItemRetainsState{no}

\objItemCompatibility{\MaxName{} 3.x and \MaxName{} 4.x \{OS 9 and OS X\}}

\objItemFat{Fat}

\objItemCommands[]

  \objListCmdBegin
  \objListCmdItem{\emph{bang}}{}
  Return the previous result, if any.
  \objListCmdEnd

\objItemFile

\objItemMessage

\objItemComments[In mathematical terms: $y_i \gets \lceil x_i \rceil$, where $x_1,x_2,\dots,x_n$ is the inlet value and
$y_1,y_2,\dots,y_n$ is the outlet result.]

\objEnd{\objNameE{Vceiling}}
