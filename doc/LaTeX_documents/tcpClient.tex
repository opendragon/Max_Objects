\ProvidesFile{tcpClient.tex}[v1.1.7]
\startObject{\objNameS{tcpClient}}{tcpClient}
\index{Themes!TCP/IP!tcpClient}
\objPicture{tcpClientsymbol.ps}
\objItemDescription{\objNameD{tcpClient} is an interface to the TCP/IP stack on a Macintosh,
providing an endpoint client to communicate with a \objReference{tcpServer} or a \objReference{tcpMultiServer}
object.}

\objItemCreated{August 1998}

\objItemVersion{1.1.7}

\objItemHelp{yes}

\objItemTheme{TCP/IP}

\objItemClass{Devices}

\objItemArgs{\nothing}

  \objListArgBegin
  \objListArgItem{ip-address}{(optional) symbol}{the address of the machine running the
       \objName{tcpServer} or \objName{tcpMultiServer} object to communicate with,
       in `dotted' notation.
       The default address is `10.11.12.13'.}

  \objListArgItem{port}{(optional) integer}{the number of the port to communicate on.
       The default port is 65535, which is also the maximum acceptable port number.}

  \objListArgItem{buffers}{(optional) integer}{the number of receive buffers to use.
      The default number of buffers is 25}
      
  \objListArgEnd

\objItemInlet{\nothing}

  \objListIOBegin
  \objListIOItem{list}{the command input}
  \objListIOEnd

\objItemOutlet{\nothing}

  \objListIOBegin
  \objListIOItem{list}{the status or response.
     Status messages (triggered by a \objCmdQ{bang} or \objCmdQ{status} command) appear as a
        two or four element list, starting with the symbol `status'; response messages appear
        as a list, starting with the symbol `reply' and `self' messages appear as a two element
        list, starting with the symbol `self'}

  \objListIOItem{bang}{an error was detected}

  \objListIOEnd

\objItemCompanion{none}

\objItemStandalone{no, works with a \objName{tcpServer} or a \objName{tcpMultiServer} object on
    the same computer or another computer that is reachable via a TCP/IP network}

\objItemRetainsState{yes}

\objItemCompatibility{\MaxName{} 3.x and \MaxName{} 4.x \{OS 9 and OS X\}}

\objItemFat{PPC-only}

\objItemCommands[]

  \objListCmdBegin

  \objListCmdItem{\emphcorr{bang}}{}
  Report the state of the communication (`unbound', `bound', `listening', `connecting',
  `connected', `disconnecting' or `unknown') as a two or five element list, with the symbol
  `status' as its first element.
  If the state is not `connected', the list will contain two elements.
  If the state is `connected', the third element is the IP address of the server, expressed as a
  symbol in `dotted' notation, the fourth element is the port of the server and the fifth element is either
  `raw' or `max'.

  \objListCmdItem{connect}{}
  Connect to the \objName{tcpServer} or \objName{tcpMultiServer} object on the machine at the
  specified address and port.

  \objListCmdItem{disconnect}{}
  Close any existing connection to a \objName{tcpServer} or a \objName{tcpMultiServer} object.

  \objListCmdItem{\emphcorr{float}}{}
  Send the given floating point value to the \objName{tcpServer} or \objName{tcpMultiServer} object.

  \objListCmdItem{\emphcorr{integer}}{}
  Send the given integer value to the \objName{tcpServer} or \objName{tcpMultiServer} object.

  \objListCmdItem{list}{anything}
  Send the given list to the \objName{tcpServer} or \objName{tcpMultiServer} object.
  \objListCmdItem{mode}{symbol}{}
  Set the operating mode of the communication to either `raw' or `max'.
  In `raw' mode, data is transferred without any structure; in `max' mode it is as described below.
  Both ends of the communication must agree on the mode.
  
  \objListCmdItem{port}{integer}
  Set the number of the port to communicate on.

  \objListCmdItem{self}{}
  Returns the IP address of this object as a two element list, with the symbol `self' as its first
  element.
  
  \objListCmdItem{send}{anything}
  Send the arguments to the \objName{tcpServer} or \objName{tcpMultiServer} object.
  This message permits the transmission of an arbitrary list; in particular any list that could be
  confused with a command to the \objNameX{tcpClient} object.

  \objListCmdItem{server}{symbol}
  Specify the address of the machine running the \objName{tcpServer} or \objName{tcpMultiServer}
  object to communicate with, in `dotted' notation.

  \objListCmdItem{status}{}
  Report the state of the communication (`unbound', `bound', `listening', `connecting', `connected',
  `disconnecting' or `unknown') as a two- or five element list, with the symbol `status' as its
  first element.
  If the state is not `connected', the list will contain two elements.
  If the state is `connected', the third element is the IP address of the server, expressed as a
  symbol in `dotted' notation, the fourth element is the port of the server and the fifth element is either
  `raw' or `max'.

  \objListCmdItem{verbose}{\textnormal{[}on\textnormal{/}off\textnormal{]}}
  Communication tracing to the \MaxName{} window will be enabled (`on'), disabled (`off') or reversed,
  if no argument is given.
  
  \objListCmdEnd

\objItemFile

\objItemMessage[]

\begin{quote}
The data sent through the TCP/IP objects is structured as follows:
\begin{enumerate}[ 1)]
\item A fixed header containing the following fields:
\begin{enumerate}[ a)]
\item The number of Atoms which follow, as a two byte number,
\item A sanity check field as a two byte number and
\item The type of Atoms to follow, as a single byte
\end{enumerate}
\item Zero or more Atoms, represented via the following fields:
\begin{enumerate}[ a)]
\item (optional) The type of the Atom, as a single byte,
\item The Atom, represented as described next.
\end{enumerate}
\end{enumerate}
All multi-byte values are sent most-significant-byte first (network byte order, which is big-endian).
The sanity check field is the negative of the sum of the total number of bytes in the data and
the number of Atoms sent.
Floating-point values are represented using IEEE 32-bit floating point.
Integers are represented as signed four byte quantities.
Symbols are represented as strings, with a leading length (a two byte number) and a trailing null
character. Note that the length includes the trailing null character, so it is always non-zero.

If all the Atoms are the same type, the type is placed in the header and the individual Atom type
fields are not transmitted.
If there is more than one type of Atom present, the Atom type field is transmitted for each Atom,
and the type in the fixed header is set to eight (8).
If the number of Atoms is zero, the type in the fixed header is also set to zero.

The type of the Atom is one of the following values:
\begin{enumerate}[ 1:]
\item Integer,
\item Floating point,
\item String,
\item[10:] Semicolon,
\item[11:] Comma or
\item[12:] Dollar
\end{enumerate}

The following sequence of values represent the list `123 14.7, Chuck':
\begin{verbatim}
04    The number of Atoms which follow
-27   The sanity check field (3 Atoms, 24 bytes)
8     The type of Atoms (eight indicates a mixture of Atoms)
1     An integer value follows
0123  The value `123' as a 32-bit integer
2     A floating-point value follows
14.70 The value '14.7' as a 32-bit floating-point number
11    A comma follows (no associated data)
3     A string value follows
06    The length of the string
Chuck The characters of the string
0     The trailing null character of the string
\end{verbatim}
The largest message that can be processed is 24,000 bytes, which could contain 6,000 integers or
floating-point values or a single string of 23,997 characters.
\end{quote}

\objItemComments[Once a communication path is established between a \objNameX{tcpClient} object and
a \objName{tcpServer} or \objName{tcpMultiServer} object, either object can send messages---the
path is full-duplex.
Figure~\objImageReference{diagram:tcpclientstate} is a state diagram for \objNameX{tcpClient} objects.]
\objDiagram{tcpClientstate.ps}{tcpclientstate}{State diagram for \objNameX{tcpClient} objects}

\objEnd{\objNameE{tcpClient}}

% $Log: tcpClient.tex,v $
% Revision 1.7  2006/07/20 04:47:54  churchoflambda
% Re-added the files to record their changes.
%
% Revision 1.5  2005/08/02 15:07:09  churchoflambda
% Added CVS tags; add rail diagrams for pfsm, map1d, map2d, map3d and listen.
%
