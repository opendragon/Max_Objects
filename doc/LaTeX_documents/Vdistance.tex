\ProvidesFile{Vdistance.tex}[v1.0.2]
\startObject{\objNameS{Vdistance}}{Vdistance}
\index{Themes!Miscellaneous!Vdistance}
\index{Themes!Vector~manipulation!Vdistance}
\index{Vectors!Monadic~operations!Vdistance}
\objPicture{Vdistancesymbol.ps}
\objItemDescription{\objNameD{Vdistance} calculates the length of its input list, considered as an
n-dimensional vector.}

\objItemCreated{April 2001}

\objItemVersion{1.0.2}

\objItemHelp{yes}

\objItemTheme{Vector manipulation}

\objItemClass{Lists}

\objItemArgs{none}

\objItemInlet{\nothing}

  \objListIOBegin
  \objListIOItem{anything}{the input to be processed}
  \objListIOEnd

\objItemOutlet{\nothing}

  \objListIOBegin
  \objListIOItem{anything}{the input after calculating the length, or the previous result
      (if a `bang' is received)}
  \objListIOEnd

\objItemCompanion{none}

\objItemStandalone{yes}

\objItemRetainsState{no}

\objItemCompatibility{\MaxName{} 3.x and \MaxName{} 4.x \{OS 9 and OS X\}}

\objItemFat{Fat}

\objItemCommands[]

  \objListCmdBegin
  \objListCmdItem{\emphcorr{bang}}{}
  Return the previous result, if any.
  \objListCmdEnd

\objItemFile

\objItemMessage

\objItemComments[In mathematical terms: $y \gets | x |$ or
$y \gets \sqrt{(\sum_{i=1}^n x_i^2)}$, where $x = x_1,x_2,\dots,x_n$ is the inlet value and
$y = y_1,y_2,\dots,y_n$ is the outlet result.]

\objEnd{\objNameE{Vdistance}}
