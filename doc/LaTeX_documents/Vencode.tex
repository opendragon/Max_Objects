\ProvidesFile{Vencode.tex}[v1.0.0]
\startObject{\objNameS{Vencode}}{Vencode}
\index{Themes!Vector~manipulation!Vencode}
\index{Vectors!Dyadic~operations!Vencode}
\objPicture{Vencodesymbol.ps}
\objItemDescription{\objNameD{Vencode} is an implementation of the \compLang{APL} `encode' operator 
($\top$)\index{APL!encode},
which is used to convert a number into an encoded representation according to a coding scheme or base.}

\objItemCreated{June 2003}

\objItemVersion{1.0.0}

\objItemHelp{yes}

\objItemTheme{Vector manipulation}

\objItemClass{Arith/Logic/Bitwise, Lists}

\objItemArgs{\ }

  \objListArgBegin
  \objListArgItem{base1}{integer}{the leftmost base to be applied.
     If this is the only base it will be repeated as often as needed to do the conversion.
     If there are more bases and this base is negative, it will be repeated (as it's absolute value) as often as needed to do the conversion,
     after the other bases have been applied.
     Bases must be greater than 1.}
  \objListArgItem{base2}{(optional) integer}{the next-to-last base to apply.}
  \objListArgItem{base3}{(optional) integer}{the next-to-next-to-last base to apply.}
  \objListArgItem{base4}{(optional) integer}{the next-to-$\ldots$-last base to apply.}
  \objListArgItem{base5}{(optional) integer}{the last base to apply.
     The input number is divided by each base, in sequence, and the remainders are collected into the output list.} 
  \objListArgEnd

\objItemInlet{\ }

  \objListIOBegin
  \objListIOItem{list\textnormal{/}number}{the number to be encoded.
  	A list results in a sequence of lists being generated.}
  \objListIOEnd

\objItemOutlet{\ }

  \objListIOBegin
  \objListIOItem{list}{the encoded result}
  \objListIOEnd

\objItemCompanion{none}

\objItemStandalone{yes}

\objItemRetainsState{no}

\objItemCompatibility{\MaxName{} 4.x \{OS 9 and OS X\}}

\objItemFat{PPC-only}

\objItemCommands

\objItemFile

\objItemMessage

\objItemComments

\objEnd{\objNameE{Vencode}}
